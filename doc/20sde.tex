\documentclass{article}
\usepackage[pagebackref,letterpaper=true,colorlinks=true,pdfpagemode=none,urlcolor=blue,linkcolor=blue,citecolor=blue,pdfstartview=FitH]{hyperref}

\usepackage{amsmath,amsfonts}
\usepackage{graphicx}
\usepackage{color}


\usepackage{algorithm}
\usepackage[noend]{algpseudocode}


\setlength{\oddsidemargin}{0pt}
\setlength{\evensidemargin}{0pt}
\setlength{\textwidth}{6.0in}
\setlength{\topmargin}{0in}
\setlength{\textheight}{8.5in}


\setlength{\parindent}{0in}
\setlength{\parskip}{5px}

%\input{macrosblog}

%%%%%%%%% For wordpress conversion

\def\more{}

\newif\ifblog
\newif\iftex
\blogfalse
\textrue


\usepackage{ulem}
\def\em{\it}
\def\emph#1{\textit{#1}}

\def\image#1#2#3{\begin{center}\includegraphics[#1pt]{#3}\end{center}}

\let\hrefnosnap=\href

\newenvironment{btabular}[1]{\begin{tabular} {#1}}{\end{tabular}}

\newenvironment{red}{\color{red}}{}
\newenvironment{green}{\color{green}}{}
\newenvironment{blue}{\color{blue}}{}

%%%%%%%%% Typesetting shortcuts

\def\B{\{0,1\}}
\def\xor{\oplus}

\def\P{{\mathbb P}}
\def\E{{\mathbb E}}
\def\var{{\bf Var}}

\def\N{{\mathbb N}}
\def\Z{{\mathbb Z}}
\def\R{{\mathbb R}}
\def\C{{\mathbb C}}
\def\Q{{\mathbb Q}}
\def\eps{{\epsilon}}

\def\bz{{\bf z}}

\def\true{{\tt true}}
\def\false{{\tt false}}

%%%%%%%%% Theorems and proofs

\newtheorem{exercise}{Exercise}
\newtheorem{theorem}{Theorem}
\newtheorem{lemma}[theorem]{Lemma}
\newtheorem{definition}[theorem]{Definition}
\newtheorem{corollary}[theorem]{Corollary}
\newtheorem{proposition}[theorem]{Proposition}
\newtheorem{example}{Example}
\newtheorem{remark}[theorem]{Remark}
\newenvironment{proof}{\noindent {\sc Proof:}}{$\Box$} %\medskip} 
%%%%%%%%% I added
\newtheorem{assumption}{Assumption}
%%%%%%%%

\begin{document}

\section{Abstract}
\begin{itemize}
 \item SDE and related financial models
 \item Euler method
\end{itemize}


\section{SDE}
\subsection{General problem}
We will consider the general d-dimensional SDE:
$$d X_{t} = b(X_{t}) dt + \sigma(X_{t}) dW_{t}, X_{0} = x_{0}$$
where $b: \mathbb R^{d} \mapsto \mathbb R^{d}$ is a smooth 
vector field  on  $\mathbb R^{d}$,
$\sigma: \mathbb R^{d} \mapsto \mathbb R^{d\times d}$ is a smooth 
matrix-valued function, $W$ is a d-dimensional standard Brownian motion, 
and $x_{0}$ is the initial d-dimensional vector. 


Some theoretical interests are the sufficient condition for the unique solvability, which can be founded in the literature.


\subsubsection{Example: 2-d SDE}
It can be written by system of two 1-d SDEs as the following:
$$
\left\{
\begin{array}
 {ll}
 d X_{1,t} = b_{1,t} dt + \sigma_{11,t}dW_{1,t} + \sigma_{12,t} dW_{2,t}, 
 & X_{1,0} = x_{1,0}\\
 d X_{2,t} = b_{2,t} dt + \sigma_{21,t}dW_{1,t} + \sigma_{22,t} dW_{2,t}, 
 & X_{2,0} = x_{2,0}
\end{array}
\right.
$$
In the above, we assume $W_{1}$ and $W_{2}$ are two independent 1-d Brownian motions.


\subsection{Financial models}

\subsubsection{Arithmetic BM}
We denote by $BM(\mu, \sigma^2)$ the dynamics
$$d X_t = \mu dt + \sigma dW_t.$$

\subsubsection{Geometric BM}
We denote by $GBM(\mu, \sigma^2)$ the dynamics
$$d X_t = \mu X_t dt + \sigma X_t dW_t.$$
Non-negativity of the GBM process is good for modeling stock price, namely BSM.
\begin{example}
\label{exm:gbm01}
Find $\log X_t$ for $X_t \sim GBM(\mu, \sigma^2)$.
\end{example}

\subsubsection{Stochastic volatility model: Local volatility}
Due to limit capacity of GBM in calibration, one can extend the asset price as
$$d S_t = \mu S_t dt + \sigma_t S_t dW_t.$$
The difference is that the volatility $\sigma_t$ is a random process and this model is classified as stochastic volatility model.

If volatility is modelled by $\sigma_t = \hat \sigma(t, S_t)$ for some deterministic function $\hat \sigma$, then it is called local volatility model, one of the most important case in stochastic volatility models.
\subsubsection{Stochastic volatility model: Heston model}
Heston model as a stochastic volatility model belongs to 2-d SDE in the above. However, the domain of the diffusion matrix $\sigma$ is not entire 2-d space. 

In the Heston model, the dynamic involves two processes $(S_{t}, \nu_{t})$.
More precisely, the asset price $S$ follows generalized geometric Brownian motion with random volatility process $\sqrt{\nu_{t}}$, i.e.  
$$d S_{t} = r S_{t} dt + \sqrt{\nu_{t}} S_{t} dW_{1,t},$$
while squared of volatility process $\nu$ follows CIR process
$$ d \nu_{t} = \kappa (\theta - \nu_{t}) dt + \xi \sqrt{\nu_{t}} (\rho dW_{1,t} + 
\bar \rho d W_{2,t})$$
with $\rho^{2} + \bar \rho^{2} = 1.$ 

\begin{itemize}
\item
Feller condition for its existence of the solution is
$$2\kappa \theta > \xi^{2}.$$
\end{itemize}


\begin{example}
\label{exm:heston01}
Our goal is to adapt the above Euler scheme to Heston model with the following parameters:
$$ S_{0} = 100, \nu(0) = .04, r = .05, \kappa = 1.2, 
\theta = .04, \xi = .3, \rho = .5.$$
The estimation of Call$(T =1, K = 100)$ is given as 10.3009, see Page 357 of \cite{Gla04}. We will use this for our comparison to our computation.
\end{example}




\subsubsection{Vasicek model}
It is a model for short rate $r_t$ given by OU process:
$$d r_t = \alpha(b - r_t) dt + \sigma dW_t.$$
In general, the zero bond price $P(0, T)$ follows
$$P(0, T) = \mathbb E[ \exp\{- \int_0^T r(u) du\} ].$$

\subsubsection{Ho-Lee model}
It is a short rate model given by
$$d r_t = g(t) dt + \sigma dW_t.$$

\subsubsection{Hull-White model}
It is short rate model, which extends Vasicek model, given by
$$dr_t = [g(t) + h(t) r_t] dt + \sigma(t) dW_t,$$
where $g, h, \sigma$ are given deterministic functions.

\begin{example}
\label{exm:hw01}\begin{itemize}
\item determine function $g, h, \sigma$ for the Vasicek model;
\item write explicit solution for HW.
\end{itemize}
\end{example}


\subsubsection{CIR model}
It is short rate of
$$d r_t = \alpha(b - r_t) dt + \sigma \sqrt{r_t} dW_t.$$
Note that, squared volatility in Heston model has the same dynamics.

\subsubsection{Affine class: Multifactor model}


\begin{example}
\label{exm:mf01}\
In Vasicek model, one can write its bond price by
$$P(t, T) = \exp\{-A(t, T) r_t+ C(t, T)\}.$$
Find $A, B$.
\end{example}
We say a short rate model is affine class model if its log bond price is affine in short rate. Hence, Vasicek model is affine class model.
Indeed, one can have affine class model in more general settings.

Let the short rate given by 
$$r_t = a^T X_t$$
where $a\in \mathbb R^d$ and $d$-factor process $X_t$ is given by $d$-dimensional OU process
$$d X_t = C(b- X_t) dt + D dW_t.$$
Then, it belongs to affine class.




\bibliographystyle{plain}
%\bibliographystyle{plainnat}
%\bibliographystyle{apalike}
\bibliography{../../refs}
\end{document}
