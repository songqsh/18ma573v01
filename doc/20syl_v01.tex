\documentclass[11pt]{amsart}
\usepackage{geometry}                % See geometry.pdf to learn the layout options. There are lots.
\geometry{letterpaper}                   % ... or a4paper or a5paper or ... 
%\geometry{landscape}                % Activate for for rotated page geometry
%\usepackage[parfill]{parskip}    % Activate to begin paragraphs with an empty line rather than an indent
\usepackage{graphicx}
\usepackage{amssymb}
\usepackage{epstopdf}
\DeclareGraphicsRule{.tif}{png}{.png}{`convert #1 `dirname #1`/`basename #1 .tif`.png}

\title{Syllabus}
\author{Computational Methods of Financial Mathematics (MA573)}
%\date{}                                           % Activate to display a given date or no date

\begin{document}
\maketitle
%\section{}
%\subsection{}

\begin{itemize}
 
 \item Course webpage: 
\begin{itemize}
 \item  https://github.com/songqsh/20s\_ma573
\end{itemize}

 \item Instructor: Qingshuo Song 
\begin{itemize}
 \item  Stratton Hall 202A (office),  508-831-6273 (phone), qsong@wpi.edu (email)
\end{itemize}
%\item TA: TBD
\item Lecture:
\begin{itemize}
 \item Thursday 5:30-8:20,  SH308
\end{itemize}

 \item Office hour: 
\begin{itemize}
 \item Monday 10:00am - 12:00pm, or by appointment
\end{itemize}

\item Course description:
\begin{itemize}
 \item Most realistic quantitative finance models are too complex to allow explicit analytic solutions and are solved by numerical computational methods. The first part of the course covers the application of finite difference methods to the partial differential equations and interest rate models arising in finance. Topics included are explicit, implicit and Crank-Nicholson finite difference schemes for fixed and free boundary value problems, their convergence and stability. The second part of the course covers Monte Carlo simulation methods, including random number generation, variance reduction techniques and the use of low discrepancy sequences, reinforcement learning. 
\end{itemize}

\item Reference:
\begin{itemize}
\item Paul Glasserman, \textit{Monte Carlo Methods in Financial Engineering}. Springer 2004. Available electronically via the Gordon library.
 \item Ali Hirsa, \textit{Computational Methods in Finance}. Chapman Hall/CRC 2013. 
\end{itemize}

\item Homework: 
%\begin{itemize}\item  
Questions will be gradually 
posted on course website. 
A student is expected the followings to get full credit in homework:
\begin{itemize}
\item The majority of homework shall be completed by Jupyter Notebook and updated to GitHub and the weblink on the completed homework shall be submitted to canvas by Tuesday of the following week.
\item Two or three students can volunteer 
at the beginning of each lecture
to give a  presentation 
on the homework solution. 
At least three presentations are needed for each student 
throughout the semester.
\end{itemize}
%\end{itemize}

\item Seminar:
We will invite speakers to give us presentation on their 
on-going research. One has to attend at least 3 talks to get full credits on seminar talk, see 

https://github.com/songqsh/songqsh.github.io/blob/master/doc/fin\_sem\_2.md



\item Gradings:
\begin{itemize}
 \item
 homework (20\%)
 \item seminar (15\%). Email me right after you attended a talk for your records.
 \item 
 90-minutes midterm (40\%), 
 \item one final project (35\%), 
\end{itemize}
Out of total 110\%, the achievement of the following total score will be sufficient for the stated letter grades:
$$A: 85\%, B: 65\%, C: 55\%.$$

\item (Tentative) important dates:
\begin{itemize}
 \item 02/27: Reading day
 \item 03/02: Midterm
 \item 04/30: Final project submission
\end{itemize}

\item Programming: 
\begin{itemize}
 \item We will use Python 3.x with Colab cloud computing platform for implementation. Although we will introduce minimal basic python concepts  in class, it is student's responsibility to master enough coding skills  through online resources to accomplish 
his/her course objective. All students need to bring their laptop to the class.

\end{itemize}





\item {Capstone Projects}: 
\begin{itemize}
 Students who want to do their capstone project in MA 573 are expected to sign up  or contact the instructor per email by  the second class. Capstone projects are undertaken in groups of 2 or 3 students and will cover methods not discussed in regular class. Capstone students are expected (1) to learn a new method, (2) apply it to a real-world problem, (3) write a report about the method and its application (approx. 8-10 pages) and (4) 
 give a presentation in class at end of the term (approx. 20 min).

\end{itemize}



\item \textbf{Writing Center}: The WPI Writing Center is a great resource to get help in writing your reports and project reports. This is in particular recommended for students for whom English is not the first language or that are aware that they have difficulties in scientific writing. Please sign up way ahead to be sure that you get an appointment.

Located on the first floor of Daniels Hall (room 116), the Writing Center is a valuable resource for helping you improve as a writer. Writing Center tutors are your peers (other undergraduate and graduate students at WPI) who are experienced writers themselves and who enjoy helping others tackle thinking and writing problems. Although a single tutoring session should never be seen as a quick fix for any writing difficulty, these sessions can help you identify your strengths and weaknesses, and teach you strategies for organizing, revising, and editing your course papers, projects, and presentations. Writing Center services are free and open to all WPI students in all classes, and tutors will happily work with you at any stage of the writing process (early brainstorming, revising a draft, polishing sentences in a final draft). Visit the Writing Center website 
{http://wpi.edu/+writing} to make an appointment.


\vspace{5pt}


\item \textbf{Students with disabilities}: If you need course adaptations or accommodations because of a disability, or if you have medical information to share with me that may impact your performance or participation in this course, please make an appointment with me as soon as possible.

If you have approved accommodations, please go to the Exam Proctoring Center (EPC) in Morgan Hall to pick up Letters of Accommodation.

If you have not already done so, students with disabilities who need to utilize accommodations in this class are encouraged to contact the Office of Disability Services (ODS) as soon as possible to ensure that such accommodations are implemented in a timely fashion. This office can be contacted via email: {DisabilityServices@wpi.edu}, via phone: (508) 831-4908, or in person: 137 Daniels Hall. 

\vspace{5pt}

\item \textbf{Academic honesty}: Each student is expected to familiarize him/herself with WPI's Academic Honesty policies which can be found at 
\begin{center}
https://www.wpi.edu/about/policies/academic-integrity 
\end{center}
All acts of fabrication, plagiarism, cheating, and facilitation will be prosecuted according to the university's policy. If you are ever unsure as to whether your intended actions are considered academically honest or not, please contact me.











\end{itemize}

\end{document}  