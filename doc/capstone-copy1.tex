\documentclass{article}
\usepackage[pagebackref,letterpaper=true,colorlinks=true,pdfpagemode=none,urlcolor=blue,linkcolor=blue,citecolor=blue,pdfstartview=FitH]{hyperref}

\usepackage{amsmath,amsfonts}
\usepackage{graphicx}
\usepackage{color}


\setlength{\oddsidemargin}{0pt}
\setlength{\evensidemargin}{0pt}
\setlength{\textwidth}{6.0in}
\setlength{\topmargin}{0in}
\setlength{\textheight}{8.5in}


\setlength{\parindent}{0in}
\setlength{\parskip}{5px}

%\input{macrosblog}

%%%%%%%%% For wordpress conversion

\def\more{}

\newif\ifblog
\newif\iftex
\blogfalse
\textrue


\usepackage{ulem}
\def\em{\it}
\def\emph#1{\textit{#1}}

\def\image#1#2#3{\begin{center}\includegraphics[#1pt]{#3}\end{center}}

\let\hrefnosnap=\href

\newenvironment{btabular}[1]{\begin{tabular} {#1}}{\end{tabular}}

\newenvironment{red}{\color{red}}{}
\newenvironment{green}{\color{green}}{}
\newenvironment{blue}{\color{blue}}{}

%%%%%%%%% Typesetting shortcuts

\def\B{\{0,1\}}
\def\xor{\oplus}

\def\P{{\mathbb P}}
\def\E{{\mathbb E}}
\def\var{{\bf Var}}

\def\N{{\mathbb N}}
\def\Z{{\mathbb Z}}
\def\R{{\mathbb R}}
\def\C{{\mathbb C}}
\def\Q{{\mathbb Q}}
\def\eps{{\epsilon}}

\def\bz{{\bf z}}

\def\true{{\tt true}}
\def\false{{\tt false}}

%%%%%%%%% Theorems and proofs

\newtheorem{exercise}{Exercise}
\newtheorem{theorem}{Theorem}
\newtheorem{lemma}[theorem]{Lemma}
\newtheorem{definition}[theorem]{Definition}
\newtheorem{corollary}[theorem]{Corollary}
\newtheorem{proposition}[theorem]{Proposition}
\newtheorem{example}{Example}
\newtheorem{remark}[theorem]{Remark}
\newenvironment{proof}{\noindent {\sc Proof:}}{$\Box$} %\medskip} 
%%%%%%%%% I added
\newtheorem{assumption}{Assumption}
%%%%%%%%

\begin{document}

\begin{center}
 MA573 - Capstone Project
\end{center}

Please address the following tasks step by step on the CEV model
	\[
	dS_t = rS_t \, dt + \sigma S_t^\beta S_t \, dW_t
	\]
	for a Brownian motion $W_t$, $\sigma > 0$ and $\beta \in [-1,0]$. 
	

\begin{itemize}
\item Design price engines for European call and put options using 
any of your favorite methods. 
\item Using at least two different methods, 
design price engine for your favorite discretely monitored 
barrier option pricing using both crude Monte Carlo method 
and  importance sampling and compare their performances. 

\item Calibrate CEV model to the 
market data of call/put option prices 
underlying some of your carefully chosen stocks.
The choice of the error function to be minimized is totally up to your taste.
For simplicity, we treat the market call/put prices are European style.

\item 
To get a sense of validity of the result, use your price engine on your calibrated model 
for the barrier option, and compare the results 
with its corresponding market price. 

\item Note that the stock and ETF put/call options (except Index options) are American style. Design price engines for American call and put options and repeat all above calibration procedure.
Compare the outcome with the one with European pricing.

\end{itemize}

Some general suggestions are given in this below:
\begin{itemize}
\item 
This project should be done independently and shall be uploaded to your github at last.

\item You can (and should)  consult the (on- and off-line) literature and cite it correctly.

\item All data sources of your market data shall be explicitly specified, for example, 
Bloomberg,  quandl, etc.

\item This should be a professional report, so the writing (English) should be up to professional standards. WPI's writing center (\url{https://www.wpi.edu/student-experience/resources/writing-center}) is a great resource to help you with this, you might reserve time in advance. 

\item
Not only code and numerical results, but also the analytical part is essential.
Literature review, theoretical background, 
mathematical derivation, and pseudo-codes must be included. 

\item {\it Performance comparison} among different pricing methods shall be up to you, but it's crucial.
This may include analysis on convergence, convergence rate, stability, etc. 
In case the mathematical rigorous proof is not available, 
 one shall demonstrate your point heuristically with some experimental evidence, 
 for instance, using computer running time, statistical estimate of variance, etc.


 \item Although specified in the above explicitly, 
 exotic option instruments and their methodologies on price engines are not limited to the above 
 recommendations, 
 if you have strong tendency to do so.
 
  
 \item You are encouraged to recover or outperform (bonus points) some numerical results from the existing literature;
 
 \item To get better performance in calibration, you
may want to try different types of error functions, or different ticks of stocks. 

\item Last but not least, if you have new findings, you shall highlight on it. 
Some hypothetical observations are given here: 
\begin{itemize}
 \item "More liquid stocks, better fit to CEV model", 
 \item 
"Financial sector stocks have better fit to CEV than energy sector stocks", 
\item 
"CEV is not effective model whenever stock market is about big shock", 
\item 
"Finite difference method outperforms Monte Carlo method",
\item 
"No boundary condition is needed if we revise the finite difference scheme", 
\item
"My choice on importance sampling is better than 
the paper XXX", 
\item 
"I provided the convergence rate of order 1/2 of my importance sampling",
\item 
"I used deep learning for importance sampling", 
\item ...
\end{itemize}

That means, your "new findings" could be some empirical conclusion 
or theoretical conclusion, as long as it has not been discussed in our class. 
But if you could replicate an idea or observation from some existing literature with proper citation, we still count this as your "new findings" .
\end{itemize}

Look forward to your new findings!



\end{document}
