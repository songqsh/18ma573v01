\documentclass{article}
\usepackage[pagebackref,letterpaper=true,colorlinks=true,pdfpagemode=none,urlcolor=blue,linkcolor=blue,citecolor=blue,pdfstartview=FitH]{hyperref}

\usepackage{amsmath,amsfonts, amssymb}
\usepackage{graphicx}
\usepackage{color}


\setlength{\oddsidemargin}{0pt}
\setlength{\evensidemargin}{0pt}
\setlength{\textwidth}{6.0in}
\setlength{\topmargin}{0in}
\setlength{\textheight}{8.5in}


\setlength{\parindent}{0in}
\setlength{\parskip}{5px}

%\input{macrosblog}

%%%%%%%%% For wordpress conversion

\def\more{}

\newif\ifblog
\newif\iftex
\blogfalse
\textrue


\usepackage{ulem}
\def\em{\it}
\def\emph#1{\textit{#1}}

\def\image#1#2#3{\begin{center}\includegraphics[#1pt]{#3}\end{center}}

\let\hrefnosnap=\href

\newenvironment{btabular}[1]{\begin{tabular} {#1}}{\end{tabular}}

\newenvironment{red}{\color{red}}{}
\newenvironment{green}{\color{green}}{}
\newenvironment{blue}{\color{blue}}{}

%%%%%%%%% Typesetting shortcuts

\def\B{\{0,1\}}
\def\xor{\oplus}

\def\P{{\mathbb P}}
\def\E{{\mathbb E}}
\def\var{{\bf Var}}

\def\N{{\mathbb N}}
\def\Z{{\mathbb Z}}
\def\R{{\mathbb R}}
\def\C{{\mathbb C}}
\def\Q{{\mathbb Q}}
\def\eps{{\epsilon}}

\def\bz{{\bf z}}

\def\true{{\tt true}}
\def\false{{\tt false}}

%%%%%%%%% Theorems and proofs

\newtheorem{exercise}{Exercise}
\newtheorem{theorem}{Theorem}
\newtheorem{lemma}[theorem]{Lemma}
\newtheorem{definition}[theorem]{Definition}
\newtheorem{corollary}[theorem]{Corollary}
\newtheorem{proposition}[theorem]{Proposition}
\newtheorem{example}{Example}
\newtheorem{remark}[theorem]{Remark}
\newenvironment{proof}{\noindent {\sc Proof:}}{$\Box$} %\medskip} 
%%%%%%%%% I added
\newtheorem{assumption}{Assumption}
%%%%%%%%

\begin{document}

\section{Abstract}
You will learn
\begin{itemize}
 \item Euler approximation for the solution of 2-d SDE
 \item We shall adapt Euler scheme for 2-d for Heston
\end{itemize}

\section{Problem}
\subsection{General problem}
We will perform Euler scheme for the general 2-d SDE to be considered is given as
$$d X_{t} = b(X_{t}) dt + \sigma(X_{t}) dW_{t}, X_{0} = x_{0}$$
where $b: \mathbb R^{2} \mapsto \mathbb R^{2}$ is a smooth 
vector field 
on  $\mathbb R^{2}$,
$\sigma: \mathbb R^{2} \mapsto \mathbb R^{2\times 2}$ is a smooth 
matrix-valued function, $W$ is a 2-d standard Brownian motion, 
and $x_{0}$ is the initial 2-d vector. It can be written by system of two 1-d SDEs as the following:
$$
\left\{
\begin{array}
 {ll}
 d X_{1,t} = b_{1,t} dt + \sigma_{11,t}dW_{1,t} + \sigma_{12,t} dW_{2,t}, 
 & X_{1,0} = x_{1,0}\\
 d X_{2,t} = b_{2,t} dt + \sigma_{21,t}dW_{1,t} + \sigma_{22,t} dW_{2,t}, 
 & X_{2,0} = x_{2,0}
\end{array}
\right.
$$
In the above, we assume $W_{1}$ and $W_{2}$ are two independent 1-d Brownian motions.

\subsection{Heston model}
Heston model as a stochastic volatility model belongs to 2-d SDE in the above. However, the domain of the diffusion matrix $\sigma$ is not entire 2-d space. 

In the Heston model, the dynamic involves two processes $(S_{t}, \nu_{t})$.
More precisely, the asset price $S$ follows generalized geometric Brownian motion with random volatility process $\sqrt{\nu_{t}}$, i.e.  
$$d S_{t} = r S_{t} dt + \sqrt{\nu_{t}} S_{t} dW_{1,t},$$
while squared of volatility process $\nu$ follows CIR process
$$ d \nu_{t} = \kappa (\theta - \nu_{t}) dt + \xi \sqrt{\nu_{t}} (\rho dW_{1,t} + 
\bar \rho d W_{2,t})$$
with $\rho^{2} + \bar \rho^{2} = 1.$ Feller condition for its existence of the solution is
$$2\kappa \theta > \xi^{2}.$$

Our goal is to adapt the above Euler scheme to Heston model with the following parameters:
$$ S_{0} = 100, \nu(0) = .04, r = .05, \kappa = 1.2, 
\theta = .04, \xi = .3, \rho = .5.$$
The estimation of Call$(T =1, K = 100)$ is given as 10.3009, see Page 357 of \cite{Gla04}. We will use this for our comparison to our computation.



\section{Analysis}
\subsection{Euler scheme}
The above SDE can be written as the following integral form:
$$
X_{t} = x_{0}  + \int_{0}^{t}  \mu(X_{s}) ds + \sigma(X_{s}) dW_{s}
$$
If we denote 
$$X_{t,s} = X_{s} - X_{t},$$
then 
$$X_{t,t+\delta} = \int_{t}^{t+\delta}  \mu(X_{s}) ds + \sigma(X_{s}) dW_{s}.$$
Ito formula says
$$
\mu(X_{s}) =
\mu(X_{t}) + \int_{t}^{s} (\mu'(X_{r}) + \frac 1 2 \mu''(X_{r})) dr + \mu'(X_{r}) \sigma(X_{r}) dW_{r}
$$
and
$$
\sigma(X_{s}) =
\sigma(X_{t}) + 
\int_{t}^{s} (\sigma'(X_{r}) + \frac 1 2 \sigma''(X_{r})) dr + \sigma'(X_{r}) \sigma(X_{r}) dW_{r}.
$$
If $|s - t| < \delta$ and $\mu, \sigma \in C_{b}^{2}$, then
\footnote{
$f(\delta) = O^{\delta^{\gamma}}$ means 
$|f(\delta)| \le K \delta^{\gamma}$
for some random variable $K$ and all $\delta \in (0, \epsilon)$.
}
$$\mu(X_{s}) = \mu(X_{t}) + O(\delta^{1/2})$$
and 
$$\sigma(X_{s}) = \sigma(X_{t}) + O(\delta^{1/2}).$$
Thus, $X_{t, t+\delta}$ can be rewritten as
$$X_{t, t+\delta} = \mu(X_{t}) \delta + \sigma(X_{t}) W_{t,t+\delta} + O(\delta).
$$
or 
$$
X_{t, t+\delta} \approx \mu(X_{t}) \delta + \sigma(X_{t}) W_{t,t+\delta}.
$$
With the fact that $W_{i \delta, (i+1)\delta} \sim \sqrt \delta Z_{i}$ are iid normal random variables, we can write the following pseudocode.
%\begin{itemize}
 %\item 

pseudocode eulder\_1d\_path(T, N):
\begin{itemize}
 \item partition $[0, T]$ equally by $\delta = T/N$;
 \item set initial $X_{0}^{\delta} = x_{0}$;
 \item For $i = 0, \ldots, N-1$,  with iid standard normal $Z_{i}$, 
\begin{itemize}
 \item perform $X_{i+1}^{\delta} = X_{i}^{\delta} + \mu(X_{i}^{\delta}) \delta + 
 \sigma(X_{i}^{\delta}) \sqrt \delta Z_{i}.$
\end{itemize}

\end{itemize}
%\end{itemize}


\subsection{Strong convergence rate}

euler\_1d\_path(T, N) gives a sequence of numbers:
$$(X_{0}^{\delta}, X_{1}^{\delta}, \ldots, X_{N}^{\delta}) := X^{\delta}.$$
To compare with continuous true path $(X_{t}: t\in [0,T])$, we first do the piecewise linear interpolation of $X^{\delta}$, that is
$$L^{\delta}_{t} = \frac{(i+1)\delta - t}{\delta} X_{i}^{\delta} 
+ \frac{t - i\delta}{\delta} X_{i+1}^{\delta}, \ \hbox{ if } 
i\delta \le t < (i+1)\delta.$$

\begin{theorem}
 RMSE of Euler approximation 
 under uniform norm has convergence order $1/2$, i.e.
 $$\mathbb E \Big [\sup_{0 \le t\le T} |X_{t} - L_{t}^{\delta}| \Big] \le K \delta^{1/2}.$$
\end{theorem}
\begin{proof}
 see Theorem 2.7.3 of \cite{Mao07}.
\end{proof}

{\bf ex.} Show that 
$$\mathbb E [ |X_{T} - L_{T}^{\delta} | ] \le K \delta^{1/2}.$$

\subsubsection{A remark on constant interpolation of Euler solution}
If we denote the piecewise constant interpolation by
$$C_{t}^{\delta} = 
X_{i}^{\delta}, \hbox{ if } i\delta \le t < (i+1) \delta, $$
then the above strong convergence fails. Let's use the following example to illustrate this issue.

Let $X = W$ be the Brownian motion itself.
Euler yields
$$C_{t}^{\delta} = W_{[t/\delta] \delta}.$$
Therefore, 
$$RMSE = 
\mathbb E  \Big [\sup_{0 \le t\le T} |X_{t} - C_{t}^{\delta}| \Big] 
= \mathbb E  \Big [\sup_{0 \le t\le T} |W_{t} - W_{[t/\delta]\delta}| \Big] 
=  \mathbb E  \Big [\sup_{i = 0, \ldots, N-1} Y_{i} \Big],
$$
where 
$$Y_{i} = \sup_{i\delta \le t < (i+1) \delta} |W_{t} - W_{i\delta}|.$$
Note $Y_{i} \ge |W_{i\delta, (i+1)\delta}| := \sqrt \delta |Z_{i}|,$ then 
$$RMSE \ge  \sqrt \delta  \mathbb E  
\Big [\sup_{i = 0, \ldots, N-1} |Z_{i}| \Big ]
> O(\delta^{1/2}).$$

\subsection{Weak convergence rate}
Given $Y^{\delta}$ and $X$, we define
$$e^{g} (\delta) = \Big| \mathbb E[ g(X_{T})] - 
\mathbb E[g (Y^{\delta}_{T})] \Big|.$$
Then, we say $Y^{\delta}_{T} $ converges to $X_{T}$ weakly if
$$\lim_{\delta\to 0} e^{g} (\delta) = 0, \ \forall g \in C_{b}.$$
We say weak convergence rae is $\gamma$, if
$$\exists K_{g} >0, \ s.t. \ e^{g}(\delta) \le K_{g} \delta^{\gamma}$$
for any $g\in C_{b}$.

\begin{theorem}
 $C^{\delta}$ covnverges to $X$ with $\gamma = 1.$
\end{theorem}
\begin{proof}
 see section 9.7 of  \cite{KP92}
\end{proof}

\bibliographystyle{plain}
%\bibliographystyle{plainnat}
%\bibliographystyle{apalike}
\bibliography{/Users/songqsh/Dropbox/R/refs}
\end{document}
