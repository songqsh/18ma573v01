\documentclass{article}
\usepackage[pagebackref,letterpaper=true,colorlinks=true,pdfpagemode=none,urlcolor=blue,linkcolor=blue,citecolor=blue,pdfstartview=FitH]{hyperref}

\usepackage{amsmath,amsfonts}
\usepackage{graphicx}
\usepackage{color}


\setlength{\oddsidemargin}{0pt}
\setlength{\evensidemargin}{0pt}
\setlength{\textwidth}{6.0in}
\setlength{\topmargin}{0in}
\setlength{\textheight}{8.5in}


\setlength{\parindent}{0in}
\setlength{\parskip}{5px}

%\input{macrosblog}

%%%%%%%%% For wordpress conversion

\def\more{}

\newif\ifblog
\newif\iftex
\blogfalse
\textrue


\usepackage{ulem}
\def\em{\it}
\def\emph#1{\textit{#1}}

\def\image#1#2#3{\begin{center}\includegraphics[#1pt]{#3}\end{center}}

\let\hrefnosnap=\href

\newenvironment{btabular}[1]{\begin{tabular} {#1}}{\end{tabular}}

\newenvironment{red}{\color{red}}{}
\newenvironment{green}{\color{green}}{}
\newenvironment{blue}{\color{blue}}{}

%%%%%%%%% Typesetting shortcuts

\def\B{\{0,1\}}
\def\xor{\oplus}

\def\P{{\mathbb P}}
\def\E{{\mathbb E}}
\def\var{{\bf Var}}

\def\N{{\mathbb N}}
\def\Z{{\mathbb Z}}
\def\R{{\mathbb R}}
\def\C{{\mathbb C}}
\def\Q{{\mathbb Q}}
\def\eps{{\epsilon}}

\def\bz{{\bf z}}

\def\true{{\tt true}}
\def\false{{\tt false}}

%%%%%%%%% Theorems and proofs

\newtheorem{exercise}{Exercise}
\newtheorem{theorem}{Theorem}
\newtheorem{lemma}[theorem]{Lemma}
\newtheorem{definition}[theorem]{Definition}
\newtheorem{corollary}[theorem]{Corollary}
\newtheorem{proposition}[theorem]{Proposition}
\newtheorem{example}{Example}
\newtheorem{remark}[theorem]{Remark}
\newenvironment{proof}{\noindent {\sc Proof:}}{$\Box$} %\medskip} 
%%%%%%%%% I added
\newtheorem{assumption}{Assumption}
%%%%%%%%

\begin{document}

\begin{center}
 MA573 - Project
\end{center}


The output should be a complete report, 
addressing the following tasks step by step.

\begin{itemize}
\item Design price engine for 
\begin{itemize}
  \item American call and put options 
  \item one of your favorite exotic options, for instance, 
  discretely monitored barrier options, asian options, 
\end{itemize}
underlying CEV model
	\[
	dS_t = rS_t \, dt + \sigma S_t^\beta S_t \, dW_t
	\]
	for a Brownian motion $W_t$, $\sigma > 0$ and $\beta \in [-1,0]$. 
\item Calibrate CEV model to the 
market data of American call/put option prices 
underlying some of your carefully chosen stocks.


\item 
To get a sense of validity of the result, calculate some exotic option prices 
using your calibrated model, and compare the results with its corresponding market price.

\end{itemize}

Some suggestions are given this below:
\begin{itemize}
\item 
This project should be done in groups of 2 (exceptionally 3) people and shall be uploaded to your github.

\item You can (and should) of course consult the (on- and off-line) literature and cite it correctly.

\item The data source of all your market data shall be explicitly specified, for example, 
Bloomberg,  quandl, etc..

\item This should be a professional report, so the writing (English) should be up to professional standards. WPI's writing center (\url{https://www.wpi.edu/student-experience/resources/writing-center}) is a great resource to help you with this, you might reserve time in advance. 

 \item Methodologies on price engines are free to choose, 
 but you may want to convince me why it is correct, efficient, or convenient, etc. 
 This includes derivation, pseudo-code, convergence, 
 convergence rate, etc. 
 In case the mathematical rigorous proof is not available, 
 one shall demonstrate your point with some numerical experiments.
 
 \item You are encouraged to recover or outperform (bonus points) some numerical results from the existing literature;
 
 \item You are encouraged to design your price engine using different methods, and compare their advantage/disadvantages; For instance, one can 
\begin{itemize}
 \item One can design American option pricing using both 
 Monte Carlo method and
various schemes of finite difference method.
\item One can design Asian option using both 
crude Monte Carlo method and control variates methods;
\item One can design barrier option pricing using both Monte Carlo method and  importance sampling.
 
 \end{itemize} 
 
\end{itemize}





\end{document}
