\documentclass{article}
\usepackage[pagebackref,letterpaper=true,colorlinks=true,pdfpagemode=none,urlcolor=blue,linkcolor=blue,citecolor=blue,pdfstartview=FitH]{hyperref}

\usepackage{amsmath,amsfonts}
\usepackage{graphicx}
\usepackage{color}


\setlength{\oddsidemargin}{0pt}
\setlength{\evensidemargin}{0pt}
\setlength{\textwidth}{6.0in}
\setlength{\topmargin}{0in}
\setlength{\textheight}{8.5in}


\setlength{\parindent}{0in}
\setlength{\parskip}{5px}

%\input{macrosblog}

%%%%%%%%% For wordpress conversion

\def\more{}

\newif\ifblog
\newif\iftex
\blogfalse
\textrue


\usepackage{ulem}
\def\em{\it}
\def\emph#1{\textit{#1}}

\def\image#1#2#3{\begin{center}\includegraphics[#1pt]{#3}\end{center}}

\let\hrefnosnap=\href

\newenvironment{btabular}[1]{\begin{tabular} {#1}}{\end{tabular}}

\newenvironment{red}{\color{red}}{}
\newenvironment{green}{\color{green}}{}
\newenvironment{blue}{\color{blue}}{}

%%%%%%%%% Typesetting shortcuts

\def\B{\{0,1\}}
\def\xor{\oplus}

\def\P{{\mathbb P}}
\def\E{{\mathbb E}}
\def\var{{\bf Var}}

\def\N{{\mathbb N}}
\def\Z{{\mathbb Z}}
\def\R{{\mathbb R}}
\def\C{{\mathbb C}}
\def\Q{{\mathbb Q}}
\def\eps{{\epsilon}}

\def\bz{{\bf z}}

\def\true{{\tt true}}
\def\false{{\tt false}}

%%%%%%%%% Theorems and proofs

\newtheorem{exercise}{Exercise}
\newtheorem{theorem}{Theorem}
\newtheorem{lemma}[theorem]{Lemma}
\newtheorem{definition}[theorem]{Definition}
\newtheorem{corollary}[theorem]{Corollary}
\newtheorem{proposition}[theorem]{Proposition}
\newtheorem{example}{Example}
\newtheorem{remark}[theorem]{Remark}
\newenvironment{proof}{\noindent {\sc Proof:}}{$\Box$} %\medskip} 
%%%%%%%%% I added
\newtheorem{assumption}{Assumption}
%%%%%%%%

\begin{document}

\begin{center}
Fourier transform in option pricing
\end{center}

\begin{abstract}
Our goal is
to use Fourier transform method for European call pricing whenever characteristic function is available for its log price.
\end{abstract}

\section{Fourier transform}
There are  different types Fourier transforms. 
If we use wiki, one definition of FT is given by the following:
\begin{equation}
\label{eq:ft01}
\hat f(u) = \int_{-\infty}^\infty e^{-2\pi ix u} f(x) dx.
\end{equation}

{\bf ex.} Write $\hat f(-u)$.

\begin{proposition} \label{p:ift01}
If $\hat f$ is FT of $f$ in the above sense, then $f$ is the inverse FT of $\hat f$ in the sense
$$f(x) = \int_{-\infty}^\infty e^{2\pi ix u} \hat f(u) du = \hat {\hat f} (-x).$$
\end{proposition}

In our context, we will use different definition of Fourier transform.
\begin{definition}
\label{d:ft01}
FT of $f$ is a function defined by
$$\mathcal F[f] (u) = \int_{-\infty}^\infty e^{ ix u} f(x) dx.$$
If $f$ is a density function of a random variable $X$, then $\mathcal F[f]$ is called characteristic function of $X$.
\end{definition}

{\bf ex.}
Prove $\mathcal F[f] (u) = \hat f(- \frac{u}{2 \pi})$.

\begin{proposition}
Inverse transform of $\mathcal F$ is given by
$$\mathcal F^{-1}[h](x) = \frac{1}{2\pi}\int_{-\infty}^\infty h(u) e^{-iux} du.$$
\end{proposition}
\begin{proof}
It's enough to show that
$$\mathcal F^{-1} \circ \mathcal F[f] = f.$$
Setting $h (u) = \mathcal F[f] (u) =  \hat f(- \frac{u}{2 \pi})$, we have
$$
\begin{array}
{ll}
\mathcal F^{-1}[h](x) &= \frac{1}{2\pi}\int_{-\infty}^\infty \hat f(- \frac{u}{2 \pi}) e^{-iux} du
\\ & = \int_{-\infty}^\infty \hat f(v) e^{2\pi i v x} dv
\end{array}
$$
and the conclusion holds by Propostion \ref{p:ift01}.
\end{proof}

{\bf ex.} If $f$ is a real valued function, then prove that
\begin{enumerate}
\item
The real part of $\mathcal F[f]$ is even,
\item The imaginary part of $\mathcal F[f]$ is odd.
\end{enumerate}


There are many useful well known Fourier transforms, see \url{https://en.wikipedia.org/wiki/Fourier_transform}.
For instance, 
$$\mathcal F[\frac 1 x](u) = i \pi \text{sgn}(u).$$
Let's use this to prove the following identity. 

{\bf ex.}
Prove $$\int_0^\infty \frac{\sin x}{x} dx.$$

{\bf Proof.}
$$
\begin{array}
{ll}
\int_0^\infty \frac{\sin x}{x} dx & = \frac 1 2 \int_{-\infty}^\infty \frac{\sin x}{x} dx
\\ & = 
\frac 1 2 \text{Img} \int_{-\infty}^\infty \frac{e^{ix}}{x} dx
\\ & = 
\frac 1 2 \text{Img} \int_{-\infty}^\infty \frac{e^{i u x}}{x} dx \Big|_{u = 1}
\\ & = 
\frac 1 2 \text{Img} \mathcal F[1/x](1) = \pi/2.
\end{array}•
$$



%\bibliographystyle{plain}
%\bibliographystyle{plainnat}
%\bibliographystyle{apalike}
%\bibliography{/Users/songqsh/Dropbox/R/refs}

\end{document}
