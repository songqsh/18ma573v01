\documentclass{article}
\usepackage[pagebackref,letterpaper=true,colorlinks=true,pdfpagemode=none,urlcolor=blue,linkcolor=blue,citecolor=blue,pdfstartview=FitH]{hyperref}

\usepackage{amsmath,amsfonts}
\usepackage{graphicx}
\usepackage{color}


\setlength{\oddsidemargin}{0pt}
\setlength{\evensidemargin}{0pt}
\setlength{\textwidth}{6.0in}
\setlength{\topmargin}{0in}
\setlength{\textheight}{8.5in}


\setlength{\parindent}{0in}
\setlength{\parskip}{5px}

%\input{macrosblog}

%%%%%%%%% For wordpress conversion

\def\more{}

\newif\ifblog
\newif\iftex
\blogfalse
\textrue


\usepackage{ulem}
\def\em{\it}
\def\emph#1{\textit{#1}}

\def\image#1#2#3{\begin{center}\includegraphics[#1pt]{#3}\end{center}}

\let\hrefnosnap=\href

\newenvironment{btabular}[1]{\begin{tabular} {#1}}{\end{tabular}}

\newenvironment{red}{\color{red}}{}
\newenvironment{green}{\color{green}}{}
\newenvironment{blue}{\color{blue}}{}

%%%%%%%%% Typesetting shortcuts

\def\B{\{0,1\}}
\def\xor{\oplus}

\def\P{{\mathbb P}}
\def\E{{\mathbb E}}
\def\var{{\bf Var}}

\def\N{{\mathbb N}}
\def\Z{{\mathbb Z}}
\def\R{{\mathbb R}}
\def\C{{\mathbb C}}
\def\Q{{\mathbb Q}}
\def\eps{{\epsilon}}

\def\bz{{\bf z}}

\def\true{{\tt true}}
\def\false{{\tt false}}

%%%%%%%%% Theorems and proofs

\newtheorem{exercise}{Exercise}
\newtheorem{theorem}{Theorem}
\newtheorem{lemma}[theorem]{Lemma}
\newtheorem{definition}[theorem]{Definition}
\newtheorem{corollary}[theorem]{Corollary}
\newtheorem{proposition}[theorem]{Proposition}
\newtheorem{example}{Example}
\newtheorem{remark}[theorem]{Remark}
\newenvironment{proof}{\noindent {\sc Proof:}}{$\Box$} %\medskip} 
%%%%%%%%% I added
\newtheorem{assumption}{Assumption}
%%%%%%%%

\begin{document}

\begin{center}
Fourier transform in option pricing
\end{center}

\begin{abstract}
Our goal is
to use Fourier transform method for European call pricing whenever characteristic function is available for its log price.
\end{abstract}

\section{Fourier transform}
There are  different types Fourier transforms. 
If we use wiki, one definition of FT is given by the following:
\begin{equation}
\label{eq:ft01}
\hat f(u) = \int_{-\infty}^\infty e^{-2\pi ix u} f(x) dx.
\end{equation}

{\bf ex.} Write $\hat f(-u)$.

\begin{proposition} \label{p:ift01}
If $\hat f$ is FT of $f$ in the above sense, then $f$ is the inverse FT of $\hat f$ in the sense
$$f(x) = \int_{-\infty}^\infty e^{2\pi ix u} \hat f(u) du = \hat {\hat f} (-x).$$
\end{proposition}

In our context, we will use different definition of Fourier transform.
\begin{definition}
\label{d:ft01}
FT of $f$ is a function defined by
$$\mathcal F[f] (u) = \int_{-\infty}^\infty e^{ ix u} f(x) dx.$$
If $f$ is a density function of a random variable $X$, then $\mathcal F[f]$ is called characteristic function of $X$.
\end{definition}

{\bf ex.}
Prove $\mathcal F[f] (u) = \hat f(- \frac{u}{2 \pi})$.

\begin{proposition}
Inverse transform of $\mathcal F$ is given by
$$\mathcal F^{-1}[h](x) = \frac{1}{2\pi}\int_{-\infty}^\infty h(u) e^{-iux} du.$$
\end{proposition}
\begin{proof}
It's enough to show that
$$\mathcal F^{-1} \circ \mathcal F[f] = f.$$
Setting $h (u) = \mathcal F[f] (u) =  \hat f(- \frac{u}{2 \pi})$, we have
$$
\begin{array}
{ll}
\mathcal F^{-1}[h](x) &= \frac{1}{2\pi}\int_{-\infty}^\infty \hat f(- \frac{u}{2 \pi}) e^{-iux} du
\\ & = \int_{-\infty}^\infty \hat f(v) e^{2\pi i v x} dv
\end{array}
$$
and the conclusion holds by Propostion \ref{p:ift01}.
\end{proof}

{\bf ex.} If $f$ is a real valued function, then prove that
\begin{enumerate}
\item
The real part of $\mathcal F[f]$ is even,
\item The imaginary part of $\mathcal F[f]$ is odd.
\end{enumerate}


There are many useful well known Fourier transforms, see \url{https://en.wikipedia.org/wiki/Fourier_transform}.
For instance, 
$$\mathcal F[\frac 1 x](u) = i \pi \text{sgn}(u).$$
Let's use this to prove the following identity. 

{\bf ex.}
Prove $$\int_0^\infty \frac{\sin x}{x} dx = \pi/2.$$

{\bf Proof.}
$$
\begin{array}
{ll}
\int_0^\infty \frac{\sin x}{x} dx & = \frac 1 2 \int_{-\infty}^\infty \frac{\sin x}{x} dx
\\ & = 
\frac 1 2 \text{Img} \int_{-\infty}^\infty \frac{e^{ix}}{x} dx
\\ & = 
\frac 1 2 \text{Img} \int_{-\infty}^\infty \frac{e^{i u x}}{x} dx \Big|_{u = 1}
\\ & = 
\frac 1 2 \text{Img} \mathcal F[1/x](1) = \pi/2.
\end{array}
$$

\section{Fourier transform methods in option pricing}
We consider the price of European call with maturity $T$ and strike $K$ underlying a given stock price $S_T$.
In many cases, the characteristic function of the log price $\ln S_T$ can be explicitly written, for instance in BSM and Heston.
Our question is: 
\begin{center}
If a characteristic function of $\ln S_T$ is given, what is the price of Call with $T$ and $K$?
\end{center}

We assume that 
\begin{itemize}
\item
interest rate is $r>0$, 
\item $s_T = \ln S_T$
\item $k = \ln K$
\item and the characteristic function of a random variable $s_T = \ln S_T$ is given by
$$\phi(u) = \mathbb E \exp(iu s_T).$$
\end{itemize}
\subsection{ First Fourier transform method}
Our objective is to design an engine with
\begin{itemize}
\item
Input: $r, \phi, T, K$
\item
output: $C = e^{-rT} \mathbb E[(S_T - K)^+]$.
\end{itemize}


\begin{proposition}
The price of Call(T, K) is
$$ C = S_0 I_1 - K e^{-rT} I_2,$$
where
$$I_1(\phi, k) = \frac 1 2 + \frac 1 \pi \int_0^\infty Re \Big( \frac{e^{-iu k} \phi(u - i)}{iu \phi(-i)} \Big) du$$
and 
$$I_2(\phi, k) = \frac 1 2 + \frac 1 \pi \int_0^\infty Re \Big( \frac{e^{-iu k} \phi(u)}{iu} \Big) du.$$
\end{proposition}

The above presentation actually gives straightforward evaluation for python as long as the characteristic function is available. One may use scipy.integrate.quad for two integrations in the formula.


\begin{proof}
One can verify step by step
\begin{itemize}
\item 
$I(a>b) = \frac 1 2 + \frac 1 \pi \int_0^\infty \frac{\sin (u(a-b))}{u} du$ from from $\int_0^\infty \frac{\sin x}{x} dx = \pi/2$ in 
the exercise in the above, 
\item
For $\phi = \phi_X$, we have
$$\mathbb P(X>H) = I_2(\phi, H)$$
and
$$\frac{\mathbb E[e^X I(X>H)]}{\mathbb E[ e^X]} = I_1(\phi, H).$$
\end{itemize}

$$C = \mathbb E [ e^{-rT} S_T I(\ln S_T > \ln K)] - K e^{-rT} \mathbb E [ I(\ln S_T > \ln K)].$$
Note that 
$$ \mathbb E[e^{-rT} S_T] = S_0.$$
Therefore, we have
$$C = S_0 
\frac{\mathbb E [ S_T I(\ln S_T > \ln K)]}{\mathbb E[S_T]} - K e^{-rT} \mathbb E [ I(\ln S_T > \ln K)].$$
Now, we conclude the result by utilizing the above lemmas with $X = \ln S_T$.

\end{proof}


\subsection{Fourier transform method by Carr-Madan}
We denote
\begin{itemize}
\item $$C_T(k) = e^{-rT} \mathbb E[(S_T - K)^+]$$
\item For some damping factor $\alpha>0$, we denote
$$c_T(k) = e^{\alpha k} C_T(k)$$
\item $$\psi = \mathcal F[c_T].$$
\end{itemize}

The key is to find a formula of $\psi_T$ in terms of $\phi$, and take inverse FT for $c_T(k)$. 
Our objective is to design an engine with
\begin{itemize}
\item
Input: $r, \phi, T, k, \alpha$
\item
output: $C_T(K)$
\end{itemize}
In theory, any $\alpha>0$ works for this pricing engine. However, the performance is dependent to the choice of $\alpha$.
Often, $\alpha$ is chosen in $[1/2, 2]$ for better performance.

\begin{proposition}
The call price is 
$$C_T(k) = \frac{e^{-\alpha k}}{\pi} \int_0^\infty e^{-i w k} \psi(w) dw,$$
where
$$\psi(w) = \frac{e^{-rT} \phi(w - (\alpha + 1) i)}{\alpha^2 + \alpha - w^2 + i (2\alpha+1)w}.$$
\end{proposition}
\begin{proof}
First we look for $\psi$. 
By definition, we have with $p_T$ the density of $s_T$
$$
\begin{array}
{ll}
\psi(w) &= 
\int_{-\infty}^{\infty} e^{iwk} e^{\alpha k} e^{-rT} \int_k^\infty (e^s - e^k) p_T(s) ds dk
\\ & = 
\int_{-\infty}^{\infty} e^{-rT} p_T(s) \int_{-\infty}^s (e^{(iw+\alpha)k + s} - e^{(iw+\alpha+1)k}) dkds
\end{array}
$$
This leads to the formula of $\psi$ after some calculations.

Next, we write $c_T$.
\begin{itemize}
\item
We can prove that $w \mapsto \psi(w) e^{-i w x}$ has even real part and odd imaginary part.
Indeed, $c_T$ is real, we have $$\psi = \text{even} + i \text{odd}.$$
Therefore, $$\psi(w) e^{-i w x} = (\text{even} + i \text{odd})(\text{even} + i \text{odd}) = \text{even} + i \text{odd}.$$
\item 
$$\begin{array}
{ll}
c_T(x) &= \mathcal F^{-1} [\psi](x) 
\\ & = 
\frac 1 {2\pi} \int_{-\infty}^\infty \psi(w) e^{-i w x} dw
\\ & = 
\frac 1 {\pi}  \int_{0}^\infty  \psi(w) e^{-i w x} dw
\end{array}
$$
\end{itemize}
Finally, we use the definition of $C_T(x) = e^{-\alpha k} c_T(k)$.
\end{proof}




%\bibliographystyle{plain}
%\bibliographystyle{plainnat}
%\bibliographystyle{apalike}
%\bibliography{/Users/songqsh/Dropbox/R/refs}

\end{document}
