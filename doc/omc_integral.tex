\documentclass{article}
\usepackage[pagebackref,letterpaper=true,colorlinks=true,pdfpagemode=none,urlcolor=blue,linkcolor=blue,citecolor=blue,pdfstartview=FitH]{hyperref}

\usepackage{amsmath,amsfonts}
\usepackage{graphicx}
\usepackage{color}


\setlength{\oddsidemargin}{0pt}
\setlength{\evensidemargin}{0pt}
\setlength{\textwidth}{6.0in}
\setlength{\topmargin}{0in}
\setlength{\textheight}{8.5in}


\setlength{\parindent}{0in}
\setlength{\parskip}{5px}

%\input{macrosblog}

%%%%%%%%% For wordpress conversion

\def\more{}

\newif\ifblog
\newif\iftex
\blogfalse
\textrue


\usepackage{ulem}
\def\em{\it}
\def\emph#1{\textit{#1}}

\def\image#1#2#3{\begin{center}\includegraphics[#1pt]{#3}\end{center}}

\let\hrefnosnap=\href

\newenvironment{btabular}[1]{\begin{tabular} {#1}}{\end{tabular}}

\newenvironment{red}{\color{red}}{}
\newenvironment{green}{\color{green}}{}
\newenvironment{blue}{\color{blue}}{}

%%%%%%%%% Typesetting shortcuts

\def\B{\{0,1\}}
\def\xor{\oplus}

\def\P{{\mathbb P}}
\def\E{{\mathbb E}}
\def\var{{\bf Var}}

\def\N{{\mathbb N}}
\def\Z{{\mathbb Z}}
\def\R{{\mathbb R}}
\def\C{{\mathbb C}}
\def\Q{{\mathbb Q}}
\def\eps{{\epsilon}}

\def\bz{{\bf z}}

\def\true{{\tt true}}
\def\false{{\tt false}}

%%%%%%%%% Theorems and proofs

\newtheorem{exercise}{Exercise}
\newtheorem{theorem}{Theorem}
\newtheorem{lemma}[theorem]{Lemma}
\newtheorem{definition}[theorem]{Definition}
\newtheorem{corollary}[theorem]{Corollary}
\newtheorem{proposition}[theorem]{Proposition}
\newtheorem{example}{Example}
\newtheorem{remark}[theorem]{Remark}
\newenvironment{proof}{\noindent {\sc Proof:}}{$\Box$} %\medskip} 
%%%%%%%%% I added
\newtheorem{assumption}{Assumption}
%%%%%%%%

\begin{document}

\section{Abstract}
You will learn
\begin{itemize}
 \item taking definite integral by ordinary Monte Carlo (OMC)
 \item exact sampling with python provided random number generators
\end{itemize}

\section{Problem}
Our goal is to compute, using OMC by exact sampling
$$\alpha = \int_0^1 h(x) dx$$
where
$$h(x) = 100 \cdot I_{(0, 1/100]}(x) + 1\cdot I_{(1/100, 1)} (x).$$
The exact value shall be 
$$\alpha = 1.98.$$

\section{Analysis}

\subsection{OMC by exact sampling}
To estimate 
$$\alpha = \mathbb E[X], \quad X \sim p(x)$$
one can use random number generator by computer (if possible)
$$\{iid \ X_i \sim p(x): i = 1, 2, \ldots, n, \} .$$
Then, one can compute the approximation of $\alpha$ by
$$\hat \alpha_n = \frac 1 n \sum_{i=1}^n X_i.$$

We say $\hat \alpha_n$ as OMC by exact sampling, since the sample $X_i$
produced by random generator has the same distribution as true distribution $X$, 
i.e. 
$$X_i \sim X, \ \forall i.$$
The properties of the OMC by exact sampling are listed below:
\begin{itemize}
 \item $X_1$ its self can be treated as an unbiased MC, because
 $$\mathbb E [X_1] = \alpha.$$
 However, MSE is big, ie.
 $$MSE(X_1) = Var(X) = \int x^2 p(x) dx.$$
 \item $\hat \alpha_n$ is consistent almost surely due by LLN, i.e.
 $$\hat \alpha_n \to \alpha, \hbox{ almost surely as } n\to \infty.$$
 Moreover, $\hat \alpha_n$ is unbiased too, and
 $$MSE(\hat \alpha_n) = Var(\hat \alpha_n) = \frac 1 n Var(X) \to 0.$$
\end{itemize}

\subsection{Evaluation of integral}
Back to our example, we write
$$\alpha = \mathbb E[X] = \mathbb E[h(Y)],$$
where $X = h(Y)$ and $Y\sim U(0,1)$.
In other words, although $X$-sampling is not directly available in python, 
one can use $U(0,1)$ random generator (see {\it numpy.random.uniform}) to produce $Y_i$, then compute $h(Y_i)$ for 
the sample $X_i$.

Pseudocode for omc\_integral(n):
\begin{itemize}
\item Generate $n$ iid samples $$\{iid \ Y_i\sim U(0,1): i = 1, 2, \ldots, n\};$$
\item Compute $n$ $X$ samples by
$$\{X_i = h(Y_i): i = 1, 2, \ldots, n\};$$
\item Take average of $X_i$'s
\end{itemize}

\section{Others}
\subsection{Project 1}
 For the problem setup given above, 
\begin{itemize}
 \item Find its convergence rate by the following procedure: \\
 Compute RMSE (root MSE) for $\hat \alpha_n$ in terms of $C n^{-\alpha}$.
 We say $\alpha$ as the convergence rate.
 \item Implement omc\_integral(n = 64).
 \item Demonstrate convergence rate numerically by doing the following:
\begin{itemize}
 \item Fix a batch number $m=100$;
 \item For $i$ in range(5, 10): 
\begin{itemize}
 \item run  $m$ times of omc\_integral($n=2^i$), store it into $\{\alpha_{ij}: j = 1, \ldots m\}.$
 \item compute standard deviation ({\it numpy.std}) of $\{\alpha_{ij}: j = 1, \ldots m\}$, save it to $\sigma_i$.
\end{itemize}
\item plot and find slope (scipy.stats.linregress)  for the data
$$\{(i, -\log_2 \sigma_i): i = 5, \ldots, 10\}.$$
\end{itemize}

\end{itemize}


\subsection{Project 2}
BS model assumes the distribution of stock as lognormal. In particular, with the parameters denoted by
\begin{itemize}
 \item  $S(0)$: The initial stock price
 \item
 $S(T)$: The stock price at $T$
 \item
 $r$: interest rate
 \item $\sigma$: volatility
\end{itemize}
the exact value of call and put prices with maturity $T$ and $K$, denoted by $C_0$ and $P_0$, are given by
$$C_0 = \mathbb E [e^{-rT} (S(T) - K)^+] = S_0  \Phi(d_1) - K e^{-rT} \Phi(d_2),$$
and 
$$P_0 = \mathbb E [e^{-rT} (S(T) - K)^-] = K e^{-rT} \Phi(- d_2) - S_0  \Phi(- d_1),$$
where $d_i$ are given as
$$d_1 = \frac{(r + \frac 1 2 \sigma^2) T - \ln \frac{K}{S_0}}{\sigma \sqrt T}, 
\quad 
d_2 = \frac{(r - \frac 1 2 \sigma^2) T - \ln \frac{K}{S_0}}{\sigma \sqrt T} = d_1 - \sigma \sqrt T.$$
With parameters $$S(0) = 100, K = 110, r = 4.75\%, \sigma = 20\%, T = 1$$
\begin{itemize}
 \item Find BS call and put price using BS formula above.
 \item Approximate BS call and put price using exact sampling omc.
 \item Compute theoretical convergence rate and demonstrate its convergence rate numerically.
\end{itemize}


\end{document}
