\documentclass{article}
\usepackage[pagebackref,letterpaper=true,colorlinks=true,pdfpagemode=none,urlcolor=blue,linkcolor=blue,citecolor=blue,pdfstartview=FitH]{hyperref}

\usepackage{amsmath,amsfonts}
\usepackage{graphicx}
\usepackage{color}


\setlength{\oddsidemargin}{0pt}
\setlength{\evensidemargin}{0pt}
\setlength{\textwidth}{6.0in}
\setlength{\topmargin}{0in}
\setlength{\textheight}{8.5in}


\setlength{\parindent}{0in}
\setlength{\parskip}{5px}

%\input{macrosblog}

%%%%%%%%% For wordpress conversion

\def\more{}

\newif\ifblog
\newif\iftex
\blogfalse
\textrue


\usepackage{ulem}
\def\em{\it}
\def\emph#1{\textit{#1}}

\def\image#1#2#3{\begin{center}\includegraphics[#1pt]{#3}\end{center}}

\let\hrefnosnap=\href

\newenvironment{btabular}[1]{\begin{tabular} {#1}}{\end{tabular}}

\newenvironment{red}{\color{red}}{}
\newenvironment{green}{\color{green}}{}
\newenvironment{blue}{\color{blue}}{}

%%%%%%%%% Typesetting shortcuts

\def\B{\{0,1\}}
\def\xor{\oplus}

\def\P{{\mathbb P}}
\def\E{{\mathbb E}}
\def\var{{\bf Var}}

\def\N{{\mathbb N}}
\def\Z{{\mathbb Z}}
\def\R{{\mathbb R}}
\def\C{{\mathbb C}}
\def\Q{{\mathbb Q}}
\def\eps{{\epsilon}}

\def\bz{{\bf z}}

\def\true{{\tt true}}
\def\false{{\tt false}}

%%%%%%%%% Theorems and proofs

\newtheorem{exercise}{Exercise}
\newtheorem{theorem}{Theorem}
\newtheorem{lemma}[theorem]{Lemma}
\newtheorem{definition}[theorem]{Definition}
\newtheorem{corollary}[theorem]{Corollary}
\newtheorem{proposition}[theorem]{Proposition}
\newtheorem{example}{Example}
\newtheorem{remark}[theorem]{Remark}
\newenvironment{proof}{\noindent {\sc Proof:}}{$\Box$} %\medskip} 
%%%%%%%%% I added
\newtheorem{assumption}{Assumption}
%%%%%%%%

\begin{document}

\section{Abstract}
We have learned OMC by exact sampling to evaluate a definite integral.
We are going to use the same techniques to evaluate BSM option prices.

\section{Project}
BSM model assumes the distribution of stock as lognormal. In particular, with the parameters denoted by
\begin{itemize}
 \item  $S(0)$: The initial stock price
 \item
 $S(T)$: The stock price at $T$
 \item
 $r$: interest rate
 \item $\sigma$: volatility
\end{itemize}
the exact value of call and put prices with maturity $T$ and $K$, denoted by $C_0$ and $P_0$, are given by
$$C_0 = \mathbb E [e^{-rT} (S(T) - K)^+] = S_0  \Phi(d_1) - K e^{-rT} \Phi(d_2),$$
and 
$$P_0 = \mathbb E [e^{-rT} (S(T) - K)^-] = K e^{-rT} \Phi(- d_2) - S_0  \Phi(- d_1),$$
where $d_i$ are given as
$$d_1 = \frac{(r + \frac 1 2 \sigma^2) T - \ln \frac{K}{S_0}}{\sigma \sqrt T}, 
\quad 
d_2 = \frac{(r - \frac 1 2 \sigma^2) T - \ln \frac{K}{S_0}}{\sigma \sqrt T} = d_1 - \sigma \sqrt T.$$
With parameters $$S(0) = 100, K = 110, r = 4.75\%, \sigma = 20\%, T = 1$$
\begin{itemize}
 \item Find BS call and put price using BS formula above.
 \item Approximate BS call and put price using exact sampling omc.
 \item Compute theoretical convergence rate and demonstrate its convergence rate numerically.
\end{itemize}


\end{document}
