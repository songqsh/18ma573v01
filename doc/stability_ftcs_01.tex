\documentclass{article}
\usepackage[pagebackref,letterpaper=true,colorlinks=true,pdfpagemode=none,urlcolor=blue,linkcolor=blue,citecolor=blue,pdfstartview=FitH]{hyperref}

\usepackage{amsmath,amsfonts}
\usepackage{graphicx}
\usepackage{color}


\setlength{\oddsidemargin}{0pt}
\setlength{\evensidemargin}{0pt}
\setlength{\textwidth}{6.0in}
\setlength{\topmargin}{0in}
\setlength{\textheight}{8.5in}


\setlength{\parindent}{0in}
\setlength{\parskip}{5px}

%\input{macrosblog}

%%%%%%%%% For wordpress conversion

\def\more{}

\newif\ifblog
\newif\iftex
\blogfalse
\textrue


\usepackage{ulem}
\def\em{\it}
\def\emph#1{\textit{#1}}

\def\image#1#2#3{\begin{center}\includegraphics[#1pt]{#3}\end{center}}

\let\hrefnosnap=\href

\newenvironment{btabular}[1]{\begin{tabular} {#1}}{\end{tabular}}

\newenvironment{red}{\color{red}}{}
\newenvironment{green}{\color{green}}{}
\newenvironment{blue}{\color{blue}}{}

%%%%%%%%% Typesetting shortcuts

\def\B{\{0,1\}}
\def\xor{\oplus}

\def\P{{\mathbb P}}
\def\E{{\mathbb E}}
\def\var{{\bf Var}}

\def\N{{\mathbb N}}
\def\Z{{\mathbb Z}}
\def\R{{\mathbb R}}
\def\C{{\mathbb C}}
\def\Q{{\mathbb Q}}
\def\eps{{\epsilon}}

\def\bz{{\bf z}}

\def\true{{\tt true}}
\def\false{{\tt false}}

%%%%%%%%% Theorems and proofs

\newtheorem{exercise}{Exercise}
\newtheorem{theorem}{Theorem}
\newtheorem{lemma}[theorem]{Lemma}
\newtheorem{definition}[theorem]{Definition}
\newtheorem{corollary}[theorem]{Corollary}
\newtheorem{proposition}[theorem]{Proposition}
\newtheorem{example}{Example}
\newtheorem{remark}[theorem]{Remark}
\newenvironment{proof}{\noindent {\sc Proof:}}{$\Box$} %\medskip} 
%%%%%%%%% I added
\newtheorem{assumption}{Assumption}
%%%%%%%%

\begin{document}

\section{Abstract}
\begin{itemize}
 \item Implement FTCS on heat equation
 \item Demonstrate instability
\end{itemize}

\section{Problem}
Demonstrate instability of FTCS solution of the heat equation
$$u_t = u_{xx}, \quad t>0, x\in \mathbb R$$
with initial data
$$
u(x, 0) = \phi(x), \quad x\in \mathbb R.
$$

{\bf parameters}

\begin{itemize}
 \item $\phi(x) = |1 - 10 x| \cdot I (|x| <0.1).$
 \item space step size $h = .2$
 \item time step size $\theta = h^2 = .04$.

\end{itemize}


\section{Analysis}

We use FTCS (Forward finite difference in time, Central finite difference in state) to solve the above heat equation. This means that we use finite difference form of
$$u_{t}(x, t) \simeq \frac{u(x, t+ \theta) - u(x, t)}{\theta} := 
\delta^{t}_{\theta} u (x, t)$$
and
$$u_{xx}(x, t) \simeq \frac{u(x+h, t) - 2u(x, t) + u(x-h, t)}{h^{2}}
:= \delta^{xx}_{h} u (x, t).$$
where $h$ and $\theta$ are some positive 
mesh size in space $h$ and in time, respectively. 

Discrete domain is accordingly a grid of
$$\{(jh, n\theta): j +1 \in \mathbb N, j\in \mathbb Z\}.$$
We denote by $u_{j}^{n}$ is the FTCS solution at grid point $(jh, n\theta)$, 
then we shall have
$$u_{t}(jh, n\theta) \simeq \frac{u_{j}^{n+1} - u_{j}^{n}}{\theta}, 
\quad u_{xx}(jh, n\theta) \simeq 
\frac{u_{j+1}^{n} - 2u_{j}^{n} + u_{j-1}^{n}}{h^{2}}.$$
Plug it into heat equation, we obtain discrete heat equation of
$$
\frac{u_{j}^{n+1} - u_{j}^{n}}{\theta} = \frac{u_{j+1}^{n} - 2u_{j}^{n} + u_{j-1}^{n}}{h^{2}}.
$$
For simplicity, we set
$$s = \frac{\theta}{h^{2}}$$
and isolate $u^{n+1}$ to the left hand side, then
\begin{equation}
\label{eq:ftcs01}
u_{j}^{n+1} = s u_{j+1}^{n} + (1-2s) u_{j}^{n} + s u_{j-1}^{n}, \quad \forall j\in \mathbb Z, n+1 \in \mathbb N. 
\end{equation}
Together with initial condition, we have
\begin{equation}
 \label{eq:ftcs00}
 u_{j}^{0} = \phi(jh), \quad \forall j\in \mathbb Z.
\end{equation}

As a summary, 
\begin{itemize}
 \item By the FTCS solution of heat equation, we mean 
 $$\{u_{j}^{n}: \forall j\in \mathbb Z, n \in \mathbb N \}$$ satisfying
 equations \eqref{eq:ftcs00} - \eqref{eq:ftcs01}.
 
 \item By $L^{\infty}$ convergence, we mean that the $L^{\infty}$ error 
 $$\epsilon_{h, \theta} = 
 \sup_{ \forall j\in \mathbb Z, n \in \mathbb N }
 | u_{j}^{n} - u(jh, n\theta) |
 $$
 goes to zero as $(h, \theta) \to (0^{+}, 0^{+})$.
 
 \item By $L^{\infty}$ stability, we mean the uniform boundedness of the numerical solution, i.e. there exists a constant $K$ such that
 $$\sup_{ \forall j\in \mathbb Z, n \in \mathbb N }
 | u_{j}^{n}| < K, \ \forall h, \theta>0.$$
\end{itemize}

For any numerical solution, our ultimate wish is to have its convergence.
To have a convergence, it is crucial to examine its stability. 
Accordingly, the number $s = \theta/h^{2}$ is defined for simplicity earlier, 
but it turns out to be crucial.

Implementing FTCS is essentially  a sequence of realization of the following , stencil (template) given by \eqref{eq:ftcs01} line by line in $n$.
$$
\begin{array}
 {ccc}
 \\
 & * & 
 \\
 \\
 \circ  & \circ & \circ
 \\
(s) &  (1-2s) &  (s)
\end{array}
$$

{\bf Pseudocode} heat\_ftcs($h, \theta$):
\begin{itemize}
 \item Set initial $\{u_{j}^{0}, \forall j \in \mathbb Z\}$ by \eqref{eq:ftcs00};
 \item For $n-1 \in \mathbb N$, do:
 $$u_{j}^{n} \implies u_{j}^{n+1}, \forall j  \in \mathbb Z \hbox{ by \eqref{eq:ftcs01}}.$$
\end{itemize}

The above pseudocode is not practical since the grid points are infinitely many. But, if the desired computation is for instance
$$\{u_{j}^{n}: j = a, a+1, \ldots, b-1, b\}$$
for some integers $n$ and $a<b$, then one can set initial on finitely many points
$$\{u_{j}^{0}, \quad j = a-n, a-n+1, \ldots, b+n-1, b+n\}.$$


\section{Numerical result}
We use hand computation to demonstrate instability with the parameters given as
\begin{itemize}
 \item $\phi(x) = |1 - 10 x| \cdot I (|x| <0.1).$
 \item space step size $h = .2$
 \item time step size $\theta = h^2 = .04$.
\end{itemize}

Note that, $s =1$ and corresponding stencil is
$$
\begin{array}
 {ccc}
 \\
 & * & 
 \\
 \\
 \circ  & \circ & \circ
 \\
(1) &  (-1) &  (1)
\end{array}
$$


One can easily figure out numerical outcomes 
$$\{u^{3}_{j}: j = -4, -3, \ldots, 3, 4\}$$
as follows.


$$
\begin{array}
 {ccccccccc}
 \\
 0 & 1 & -3 & 6 & -7 & 6 & -3 & 1 & 0
 \\
 \\
 & 0 & 1 & -2 & 3 & -2 & 1& 0 & 
 \\
 \\
 && 0 & 1 & -1 & 1 & 0
 \\
 \\
&&& 0 & 1 & 0
\\
&&&& (u_{0}^{0})
\end{array}
$$

It can be seen that $|u_{0}^{n}| \to \infty$ as $n\to \infty$, which demonstrates its instability.










\end{document}
