\documentclass{article}
\usepackage[pagebackref,letterpaper=true,colorlinks=true,pdfpagemode=none,urlcolor=blue,linkcolor=blue,citecolor=blue,pdfstartview=FitH]{hyperref}

\usepackage{amsmath,amsfonts}
\usepackage{graphicx}
\usepackage{color}


\setlength{\oddsidemargin}{0pt}
\setlength{\evensidemargin}{0pt}
\setlength{\textwidth}{6.0in}
\setlength{\topmargin}{0in}
\setlength{\textheight}{8.5in}


\setlength{\parindent}{0in}
\setlength{\parskip}{5px}

%\input{macrosblog}

%%%%%%%%% For wordpress conversion

\def\more{}

\newif\ifblog
\newif\iftex
\blogfalse
\textrue


\usepackage{ulem}
\def\em{\it}
\def\emph#1{\textit{#1}}

\def\image#1#2#3{\begin{center}\includegraphics[#1pt]{#3}\end{center}}

\let\hrefnosnap=\href

\newenvironment{btabular}[1]{\begin{tabular} {#1}}{\end{tabular}}

\newenvironment{red}{\color{red}}{}
\newenvironment{green}{\color{green}}{}
\newenvironment{blue}{\color{blue}}{}

%%%%%%%%% Typesetting shortcuts

\def\B{\{0,1\}}
\def\xor{\oplus}

\def\P{{\mathbb P}}
\def\E{{\mathbb E}}
\def\var{{\bf Var}}

\def\N{{\mathbb N}}
\def\Z{{\mathbb Z}}
\def\R{{\mathbb R}}
\def\C{{\mathbb C}}
\def\Q{{\mathbb Q}}
\def\eps{{\epsilon}}

\def\bz{{\bf z}}

\def\true{{\tt true}}
\def\false{{\tt false}}

%%%%%%%%% Theorems and proofs

\newtheorem{exercise}{Exercise}
\newtheorem{theorem}{Theorem}
\newtheorem{lemma}[theorem]{Lemma}
\newtheorem{definition}[theorem]{Definition}
\newtheorem{corollary}[theorem]{Corollary}
\newtheorem{proposition}[theorem]{Proposition}
\newtheorem{example}{Example}
\newtheorem{remark}[theorem]{Remark}
\newenvironment{proof}{\noindent {\sc Proof:}}{$\Box$} %\medskip} 
%%%%%%%%% I added
\newtheorem{assumption}{Assumption}
%%%%%%%%

\begin{document}


\section{Abstract}
\begin{itemize}
 \item Derive stability condition on  FTCS solution of heat equation;
 \item Give Von-Neumann condition. (no proof)
\end{itemize}

\section{Problem}
We already observed the danger of FTCS scheme for the heat equation.
Find stability condition of FTCS solution for the heat equation
$$u_t = u_{xx}, \quad t>0, x\in \mathbb R$$
with initial data
$$
u(x, 0) = \phi(x), \quad x\in \mathbb R.
$$




\section{Analysis}
\subsection{Setup}
First, we recall FTCS scheme. FTCS means that we use finite difference form of
$$u_{t}(x, t) \simeq \frac{u(x, t+ \theta) - u(x, t)}{\theta} := 
\delta^{t}_{\theta} u (x, t)$$
and
$$u_{xx}(x, t) \simeq \frac{u(x+h, t) - 2u(x, t) + u(x-h, t)}{h^{2}}
:= \delta^{xx}_{h} u (x, t).$$
where $h$ and $\theta$ are some positive 
mesh size in space $h$ and in time, respectively. 

Discrete domain is accordingly a grid of
$$\{(jh, n\theta): j +1 \in \mathbb N, j\in \mathbb Z\}.$$
We denote by $u_{j}^{n}$ is the FTCS solution at a grid point $(jh, n\theta)$, 
then we shall have
$$u_{t}(jh, n\theta) \simeq \frac{u_{j}^{n+1} - u_{j}^{n}}{\theta}, 
\quad u_{xx}(jh, n\theta) \simeq 
\frac{u_{j+1}^{n} - 2u_{j}^{n} + u_{j-1}^{n}}{h^{2}}.$$
Plug it into the heat equation, we obtain discrete heat equation 
\begin{equation}
\label{eq:ftcs01}
u_{j}^{n+1} = s u_{j+1}^{n} + (1-2s) u_{j}^{n} + s u_{j-1}^{n}, \quad \forall j\in \mathbb Z, n+1 \in \mathbb N. 
\end{equation}
with the initial condition
\begin{equation}
 \label{eq:ftcs00}
 u_{j}^{0} = \phi(jh), \quad \forall j\in \mathbb Z.
\end{equation}
where 
$$s = \frac{\theta}{h^{2}}.$$
By the FTCS solution of heat equation, we mean 
 $$\{u_{j}^{n}: \forall j\in \mathbb Z, n \in \mathbb N \}$$ 
 satisfying
 equations \eqref{eq:ftcs00} - \eqref{eq:ftcs01}.

\subsection{Solution}

In this below, we solve for FTCS solution in two steps. 
First, we 
use the technic of the separation in variable to 
find all possible
solutions satisfying \eqref{eq:ftcs01}. 
Second step is to choose specific solution by fitting the initial condition \eqref{eq:ftcs00}.

%\subsubsection{Solution}
We first search for the solution of \eqref{eq:ftcs01} given by the product of 
$j$-function and $n$-function:
$$u_{j}^{n} = X_{j} T_{n}.$$
Of course, any linear combination of such solutions shall give another solution of \eqref{eq:ftcs01}.

Plug above form into \eqref{eq:ftcs01}, it writes
$$\frac{T_{n+1}}{T_{n}} = 1 - 2s + s \frac{X_{j+1} - X_{j}}{X_{j}}.$$
Note that, left hand side is a function of $n$ while right hand side is a function of $j$ for all $(n, j)$. Thus, to be equal, they must be equal to a constant, say $\xi$, i.e.
$$\frac{T_{n+1}}{T_{n}} = 1 - 2s + s \frac{X_{j+1} - X_{j}}{X_{j}}: = \xi.$$
Therefore, we have 
$$T_{n} = \xi^{n} T_{0} = \xi^{n}, \forall n\in \mathbb N$$
and
$$
1 - 2s + s \frac{X_{j+1} - X_{j}}{X_{j}}: = \xi, \forall j\in \mathbb Z.
$$
In the above, $T_{0} =1$ is assumed w.l.o.g. (why?) Another trick is to postulate $X$ in the form of
$$
X_{j} = (e^{ikh})^{j}
$$
for some $k\in \mathbb Z$. The reason is that, we expect the solution $X$ is in $L^{2}$, and our solution could be linear combination of the above Fourier basis functions.

Then, we can solve for $\xi$ by plugging into $X$-equation above, that
$$
\xi(k) := \xi = 1 - 2s + 2s \cos khj.
$$
At last, we have general representation of the numerical solution:
$$u_{j}^{n} =  \sum_{k\in \mathbb Z} b_{k} e^{ikhj} \xi^{n}(k).$$
The rest is to determine coefficients 
$(b_{k}: k\in \mathbb Z)$ from the initial condition using orthorgonal basis functions. 

\subsection{Stability condition}
From the above solution representation, we shall have 
\begin{center}
if $|\xi(k)| \le 1$ for all $k\in \mathbb Z$, then FTCS is stable.
\end{center}
Since $s>0$, the inequality $\xi(k) \le 1$ holds automatically.
The lower bound of $\xi(k)$ is $1-4s$. So we shall require
$-1 \le 1- 4s $
for its stability, which at last gives the {\bf sufficient condition of the stability} 
by
$$s \le 1/2.$$

The example discussed above indicates that the general procedure to determine stability in a diffusion or wave problem is to separate variables in the difference equation. For the time factor we obtain a simple equation with an amplication factor $\xi(k)$. 
In the analysis above, we use $|\xi(k)| \le 1$ for the stability. More precisely, it can be shown that the correct condition necessary for stability is
$$|\xi(k)| \le 1 + O(\theta), \forall k\in \mathbb Z.$$
This is the {\bf von Neumann stability condition}.


\end{document}
