\documentclass{article}
\usepackage[pagebackref,letterpaper=true,colorlinks=true,pdfpagemode=none,urlcolor=blue,linkcolor=blue,citecolor=blue,pdfstartview=FitH]{hyperref}

\usepackage{amsmath,amsfonts}
\usepackage{graphicx}
\usepackage{color}


\setlength{\oddsidemargin}{0pt}
\setlength{\evensidemargin}{0pt}
\setlength{\textwidth}{6.0in}
\setlength{\topmargin}{0in}
\setlength{\textheight}{8.5in}


\setlength{\parindent}{0in}
\setlength{\parskip}{5px}

%\input{macrosblog}

%%%%%%%%% For wordpress conversion

\def\more{}

\newif\ifblog
\newif\iftex
\blogfalse
\textrue


\usepackage{ulem}
\def\em{\it}
\def\emph#1{\textit{#1}}

\def\image#1#2#3{\begin{center}\includegraphics[#1pt]{#3}\end{center}}

\let\hrefnosnap=\href

\newenvironment{btabular}[1]{\begin{tabular} {#1}}{\end{tabular}}

\newenvironment{red}{\color{red}}{}
\newenvironment{green}{\color{green}}{}
\newenvironment{blue}{\color{blue}}{}

%%%%%%%%% Typesetting shortcuts

\def\B{\{0,1\}}
\def\xor{\oplus}

\def\P{{\mathbb P}}
\def\E{{\mathbb E}}
\def\var{{\bf Var}}

\def\N{{\mathbb N}}
\def\Z{{\mathbb Z}}
\def\R{{\mathbb R}}
\def\C{{\mathbb C}}
\def\Q{{\mathbb Q}}
\def\eps{{\epsilon}}

\def\bz{{\bf z}}

\def\true{{\tt true}}
\def\false{{\tt false}}

%%%%%%%%% Theorems and proofs

\newtheorem{exercise}{Exercise}
\newtheorem{theorem}{Theorem}
\newtheorem{lemma}[theorem]{Lemma}
\newtheorem{definition}[theorem]{Definition}
\newtheorem{corollary}[theorem]{Corollary}
\newtheorem{proposition}[theorem]{Proposition}
\newtheorem{example}{Example}
\newtheorem{remark}[theorem]{Remark}
\newenvironment{proof}{\noindent {\sc Proof:}}{$\Box$} %\medskip} 
%%%%%%%%% I added
\newtheorem{assumption}{Assumption}
%%%%%%%%

\begin{document}


\section{Abstract}
\begin{itemize}
\item derive Crank-Nicolson scheme
\item prove unconditional stability
\end{itemize}

\section{Problem}
We have seen that FTCS scheme is stable for the heat equation
$$u_t = u_{xx}, \quad t>0, x\in \mathbb R$$
with initial data
$$
u(x, 0) = \phi(x), \quad x\in \mathbb R.
$$
when $s= \frac{\theta}{h^{2}} < 1/2$ holds.
Next, we are going to present Crank-Nicolson scheme and investigate its sability.




\section{Analysis}
\subsection{Solution}
We recall that FFD in time is
$$u_{t}(x, t) \simeq \frac{u(x, t+ \theta) - u(x, t)}{\theta} := 
\delta^{t}_{\theta} u (x, t)$$
and CFD2 in state is
$$u_{xx}(x, t) \simeq \frac{u(x+h, t) - 2u(x, t) + u(x-h, t)}{h^{2}}
:= \delta^{xx}_{h} u (x, t),$$
where $h$ and $\theta$ are some positive 
mesh size in space $h$ and in time, respectively. Again, we set
$$s = \frac{\theta}{h^{2}}.$$

Discrete domain is accordingly a grid of
$$\{(jh, n\theta): j +1 \in \mathbb N, j\in \mathbb Z\}.$$
Recall that FTCS is to find numerical values
$u_{j}^{n}$ at a grid point $(jh, n\theta)$, 
such that
$$\delta_{\theta}^{t} u(jh, n\theta) \simeq 
\frac{u_{j}^{n+1} - u_{j}^{n}}{\theta} := (\delta^{t}_{\theta} u)_{j}^{n}, 
\quad 
\delta^{xx}_{h}u(jh, n\theta) \simeq 
\frac{u_{j+1}^{n} - 2u_{j}^{n} + u_{j-1}^{n}}{h^{2}} 
:=( \delta^{xx}_{h} u)_{j}^{n}.$$
Plug it into the heat equation, we obtain FTCS discrete heat equation 
\begin{equation}
\label{eq:ftcs01}
u_{j}^{n+1} = s u_{j+1}^{n} + (1-2s) u_{j}^{n} + s u_{j-1}^{n}, \quad \forall j\in \mathbb Z, n+1 \in \mathbb N. 
\end{equation}
The Crank-Nicolson with parameter $\lambda$ (denoted by CS-$\lambda$) is 
to find numerical values
$u_{j}^{n}$ at each grid point $(jh, n\theta)$, 
such that
$$\delta_{\theta}^{t} u(jh, n\theta) \simeq  (\delta^{t}_{\theta} u)_{j}^{n}, 
\quad 
\delta^{xx}_{h}u(jh, n\theta) \simeq 
(1 - \lambda)( \delta^{xx}_{h} u)_{j}^{n} + \lambda ( \delta^{xx} u)^{n+1}_{j}.$$
Note that, if $\lambda =0$, then CS-$\lambda$ is FTCS discussed earlier.
If $\lambda = 1$, then we call CS-$\lambda$ as BTCS (Backward in time and central in state). In this below, we only discuss for $\lambda \in (0,1]$.
Plug it into the heat equation, we obtain 
CS-$\lambda$ discrete heat equation 
$$(\delta^{t}_{\theta} u)_{j}^{n} = (1 - \lambda)( \delta^{xx}_{h} u)_{j}^{n} + \lambda ( \delta^{xx} u)^{n+1}_{j},$$
which is equivalent to
\begin{equation}
 \label{eq:cs01}
 -s \lambda u_{j+1}^{n+1} + (1 + 2s\lambda) u_{j}^{n+1} - s\lambda u_{j-1}^{n+1} = s(1 - \lambda) u_{j+1}^{n} + (1 - 2s + 2s\lambda) u_{j}^{n} + 
 s(1 - \lambda) u_{j-1}^{n}.
 \end{equation}
The numerical solution is to find $(u_{j}^{n})$ satisfying \eqref{eq:cs01}   ogether with the initial condition
\begin{equation}
 \label{eq:ftcs00}
 u_{j}^{0} = \phi(jh), \quad \forall j\in \mathbb Z.
\end{equation}

 

\subsection{Stability}
 

Again by using separation in variables, we deduce that
$$T_{n} = \xi^{n}(k), X_{j} = e^{ijk h}$$
where 
\begin{equation}
 \label{eq:xi}
 \xi(k) = \frac{1 - 2(1-\lambda) s(1 - \cos k h)}{1+2\lambda s(1 - \cos k h)}.
\end{equation}




\begin{proposition}\label{p:satiblity01}
 If $\lambda \in [1/2, 1]$, then CS-$\lambda$ is always stable. In other words, there is no restriction on the selection of $(h, \theta)$.
\end{proposition}




{\bf ex.} Write stencil and pseudocode with $\lambda = 1/2$. Explain why it is an implicit scheme while FTCS is explicit scheme.



{\bf ex.} Prove the above Proposition \ref{p:satiblity01}.
\end{document}


