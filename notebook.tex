
% Default to the notebook output style

    


% Inherit from the specified cell style.




    
\documentclass[11pt]{article}

    
    
    \usepackage[T1]{fontenc}
    % Nicer default font (+ math font) than Computer Modern for most use cases
    \usepackage{mathpazo}

    % Basic figure setup, for now with no caption control since it's done
    % automatically by Pandoc (which extracts ![](path) syntax from Markdown).
    \usepackage{graphicx}
    % We will generate all images so they have a width \maxwidth. This means
    % that they will get their normal width if they fit onto the page, but
    % are scaled down if they would overflow the margins.
    \makeatletter
    \def\maxwidth{\ifdim\Gin@nat@width>\linewidth\linewidth
    \else\Gin@nat@width\fi}
    \makeatother
    \let\Oldincludegraphics\includegraphics
    % Set max figure width to be 80% of text width, for now hardcoded.
    \renewcommand{\includegraphics}[1]{\Oldincludegraphics[width=.8\maxwidth]{#1}}
    % Ensure that by default, figures have no caption (until we provide a
    % proper Figure object with a Caption API and a way to capture that
    % in the conversion process - todo).
    \usepackage{caption}
    \DeclareCaptionLabelFormat{nolabel}{}
    \captionsetup{labelformat=nolabel}

    \usepackage{adjustbox} % Used to constrain images to a maximum size 
    \usepackage{xcolor} % Allow colors to be defined
    \usepackage{enumerate} % Needed for markdown enumerations to work
    \usepackage{geometry} % Used to adjust the document margins
    \usepackage{amsmath} % Equations
    \usepackage{amssymb} % Equations
    \usepackage{textcomp} % defines textquotesingle
    % Hack from http://tex.stackexchange.com/a/47451/13684:
    \AtBeginDocument{%
        \def\PYZsq{\textquotesingle}% Upright quotes in Pygmentized code
    }
    \usepackage{upquote} % Upright quotes for verbatim code
    \usepackage{eurosym} % defines \euro
    \usepackage[mathletters]{ucs} % Extended unicode (utf-8) support
    \usepackage[utf8x]{inputenc} % Allow utf-8 characters in the tex document
    \usepackage{fancyvrb} % verbatim replacement that allows latex
    \usepackage{grffile} % extends the file name processing of package graphics 
                         % to support a larger range 
    % The hyperref package gives us a pdf with properly built
    % internal navigation ('pdf bookmarks' for the table of contents,
    % internal cross-reference links, web links for URLs, etc.)
    \usepackage{hyperref}
    \usepackage{longtable} % longtable support required by pandoc >1.10
    \usepackage{booktabs}  % table support for pandoc > 1.12.2
    \usepackage[inline]{enumitem} % IRkernel/repr support (it uses the enumerate* environment)
    \usepackage[normalem]{ulem} % ulem is needed to support strikethroughs (\sout)
                                % normalem makes italics be italics, not underlines
    

    
    
    % Colors for the hyperref package
    \definecolor{urlcolor}{rgb}{0,.145,.698}
    \definecolor{linkcolor}{rgb}{.71,0.21,0.01}
    \definecolor{citecolor}{rgb}{.12,.54,.11}

    % ANSI colors
    \definecolor{ansi-black}{HTML}{3E424D}
    \definecolor{ansi-black-intense}{HTML}{282C36}
    \definecolor{ansi-red}{HTML}{E75C58}
    \definecolor{ansi-red-intense}{HTML}{B22B31}
    \definecolor{ansi-green}{HTML}{00A250}
    \definecolor{ansi-green-intense}{HTML}{007427}
    \definecolor{ansi-yellow}{HTML}{DDB62B}
    \definecolor{ansi-yellow-intense}{HTML}{B27D12}
    \definecolor{ansi-blue}{HTML}{208FFB}
    \definecolor{ansi-blue-intense}{HTML}{0065CA}
    \definecolor{ansi-magenta}{HTML}{D160C4}
    \definecolor{ansi-magenta-intense}{HTML}{A03196}
    \definecolor{ansi-cyan}{HTML}{60C6C8}
    \definecolor{ansi-cyan-intense}{HTML}{258F8F}
    \definecolor{ansi-white}{HTML}{C5C1B4}
    \definecolor{ansi-white-intense}{HTML}{A1A6B2}

    % commands and environments needed by pandoc snippets
    % extracted from the output of `pandoc -s`
    \providecommand{\tightlist}{%
      \setlength{\itemsep}{0pt}\setlength{\parskip}{0pt}}
    \DefineVerbatimEnvironment{Highlighting}{Verbatim}{commandchars=\\\{\}}
    % Add ',fontsize=\small' for more characters per line
    \newenvironment{Shaded}{}{}
    \newcommand{\KeywordTok}[1]{\textcolor[rgb]{0.00,0.44,0.13}{\textbf{{#1}}}}
    \newcommand{\DataTypeTok}[1]{\textcolor[rgb]{0.56,0.13,0.00}{{#1}}}
    \newcommand{\DecValTok}[1]{\textcolor[rgb]{0.25,0.63,0.44}{{#1}}}
    \newcommand{\BaseNTok}[1]{\textcolor[rgb]{0.25,0.63,0.44}{{#1}}}
    \newcommand{\FloatTok}[1]{\textcolor[rgb]{0.25,0.63,0.44}{{#1}}}
    \newcommand{\CharTok}[1]{\textcolor[rgb]{0.25,0.44,0.63}{{#1}}}
    \newcommand{\StringTok}[1]{\textcolor[rgb]{0.25,0.44,0.63}{{#1}}}
    \newcommand{\CommentTok}[1]{\textcolor[rgb]{0.38,0.63,0.69}{\textit{{#1}}}}
    \newcommand{\OtherTok}[1]{\textcolor[rgb]{0.00,0.44,0.13}{{#1}}}
    \newcommand{\AlertTok}[1]{\textcolor[rgb]{1.00,0.00,0.00}{\textbf{{#1}}}}
    \newcommand{\FunctionTok}[1]{\textcolor[rgb]{0.02,0.16,0.49}{{#1}}}
    \newcommand{\RegionMarkerTok}[1]{{#1}}
    \newcommand{\ErrorTok}[1]{\textcolor[rgb]{1.00,0.00,0.00}{\textbf{{#1}}}}
    \newcommand{\NormalTok}[1]{{#1}}
    
    % Additional commands for more recent versions of Pandoc
    \newcommand{\ConstantTok}[1]{\textcolor[rgb]{0.53,0.00,0.00}{{#1}}}
    \newcommand{\SpecialCharTok}[1]{\textcolor[rgb]{0.25,0.44,0.63}{{#1}}}
    \newcommand{\VerbatimStringTok}[1]{\textcolor[rgb]{0.25,0.44,0.63}{{#1}}}
    \newcommand{\SpecialStringTok}[1]{\textcolor[rgb]{0.73,0.40,0.53}{{#1}}}
    \newcommand{\ImportTok}[1]{{#1}}
    \newcommand{\DocumentationTok}[1]{\textcolor[rgb]{0.73,0.13,0.13}{\textit{{#1}}}}
    \newcommand{\AnnotationTok}[1]{\textcolor[rgb]{0.38,0.63,0.69}{\textbf{\textit{{#1}}}}}
    \newcommand{\CommentVarTok}[1]{\textcolor[rgb]{0.38,0.63,0.69}{\textbf{\textit{{#1}}}}}
    \newcommand{\VariableTok}[1]{\textcolor[rgb]{0.10,0.09,0.49}{{#1}}}
    \newcommand{\ControlFlowTok}[1]{\textcolor[rgb]{0.00,0.44,0.13}{\textbf{{#1}}}}
    \newcommand{\OperatorTok}[1]{\textcolor[rgb]{0.40,0.40,0.40}{{#1}}}
    \newcommand{\BuiltInTok}[1]{{#1}}
    \newcommand{\ExtensionTok}[1]{{#1}}
    \newcommand{\PreprocessorTok}[1]{\textcolor[rgb]{0.74,0.48,0.00}{{#1}}}
    \newcommand{\AttributeTok}[1]{\textcolor[rgb]{0.49,0.56,0.16}{{#1}}}
    \newcommand{\InformationTok}[1]{\textcolor[rgb]{0.38,0.63,0.69}{\textbf{\textit{{#1}}}}}
    \newcommand{\WarningTok}[1]{\textcolor[rgb]{0.38,0.63,0.69}{\textbf{\textit{{#1}}}}}
    
    
    % Define a nice break command that doesn't care if a line doesn't already
    % exist.
    \def\br{\hspace*{\fill} \\* }
    % Math Jax compatability definitions
    \def\gt{>}
    \def\lt{<}
    % Document parameters
    \title{README\_v01}
    
    
    

    % Pygments definitions
    
\makeatletter
\def\PY@reset{\let\PY@it=\relax \let\PY@bf=\relax%
    \let\PY@ul=\relax \let\PY@tc=\relax%
    \let\PY@bc=\relax \let\PY@ff=\relax}
\def\PY@tok#1{\csname PY@tok@#1\endcsname}
\def\PY@toks#1+{\ifx\relax#1\empty\else%
    \PY@tok{#1}\expandafter\PY@toks\fi}
\def\PY@do#1{\PY@bc{\PY@tc{\PY@ul{%
    \PY@it{\PY@bf{\PY@ff{#1}}}}}}}
\def\PY#1#2{\PY@reset\PY@toks#1+\relax+\PY@do{#2}}

\expandafter\def\csname PY@tok@w\endcsname{\def\PY@tc##1{\textcolor[rgb]{0.73,0.73,0.73}{##1}}}
\expandafter\def\csname PY@tok@c\endcsname{\let\PY@it=\textit\def\PY@tc##1{\textcolor[rgb]{0.25,0.50,0.50}{##1}}}
\expandafter\def\csname PY@tok@cp\endcsname{\def\PY@tc##1{\textcolor[rgb]{0.74,0.48,0.00}{##1}}}
\expandafter\def\csname PY@tok@k\endcsname{\let\PY@bf=\textbf\def\PY@tc##1{\textcolor[rgb]{0.00,0.50,0.00}{##1}}}
\expandafter\def\csname PY@tok@kp\endcsname{\def\PY@tc##1{\textcolor[rgb]{0.00,0.50,0.00}{##1}}}
\expandafter\def\csname PY@tok@kt\endcsname{\def\PY@tc##1{\textcolor[rgb]{0.69,0.00,0.25}{##1}}}
\expandafter\def\csname PY@tok@o\endcsname{\def\PY@tc##1{\textcolor[rgb]{0.40,0.40,0.40}{##1}}}
\expandafter\def\csname PY@tok@ow\endcsname{\let\PY@bf=\textbf\def\PY@tc##1{\textcolor[rgb]{0.67,0.13,1.00}{##1}}}
\expandafter\def\csname PY@tok@nb\endcsname{\def\PY@tc##1{\textcolor[rgb]{0.00,0.50,0.00}{##1}}}
\expandafter\def\csname PY@tok@nf\endcsname{\def\PY@tc##1{\textcolor[rgb]{0.00,0.00,1.00}{##1}}}
\expandafter\def\csname PY@tok@nc\endcsname{\let\PY@bf=\textbf\def\PY@tc##1{\textcolor[rgb]{0.00,0.00,1.00}{##1}}}
\expandafter\def\csname PY@tok@nn\endcsname{\let\PY@bf=\textbf\def\PY@tc##1{\textcolor[rgb]{0.00,0.00,1.00}{##1}}}
\expandafter\def\csname PY@tok@ne\endcsname{\let\PY@bf=\textbf\def\PY@tc##1{\textcolor[rgb]{0.82,0.25,0.23}{##1}}}
\expandafter\def\csname PY@tok@nv\endcsname{\def\PY@tc##1{\textcolor[rgb]{0.10,0.09,0.49}{##1}}}
\expandafter\def\csname PY@tok@no\endcsname{\def\PY@tc##1{\textcolor[rgb]{0.53,0.00,0.00}{##1}}}
\expandafter\def\csname PY@tok@nl\endcsname{\def\PY@tc##1{\textcolor[rgb]{0.63,0.63,0.00}{##1}}}
\expandafter\def\csname PY@tok@ni\endcsname{\let\PY@bf=\textbf\def\PY@tc##1{\textcolor[rgb]{0.60,0.60,0.60}{##1}}}
\expandafter\def\csname PY@tok@na\endcsname{\def\PY@tc##1{\textcolor[rgb]{0.49,0.56,0.16}{##1}}}
\expandafter\def\csname PY@tok@nt\endcsname{\let\PY@bf=\textbf\def\PY@tc##1{\textcolor[rgb]{0.00,0.50,0.00}{##1}}}
\expandafter\def\csname PY@tok@nd\endcsname{\def\PY@tc##1{\textcolor[rgb]{0.67,0.13,1.00}{##1}}}
\expandafter\def\csname PY@tok@s\endcsname{\def\PY@tc##1{\textcolor[rgb]{0.73,0.13,0.13}{##1}}}
\expandafter\def\csname PY@tok@sd\endcsname{\let\PY@it=\textit\def\PY@tc##1{\textcolor[rgb]{0.73,0.13,0.13}{##1}}}
\expandafter\def\csname PY@tok@si\endcsname{\let\PY@bf=\textbf\def\PY@tc##1{\textcolor[rgb]{0.73,0.40,0.53}{##1}}}
\expandafter\def\csname PY@tok@se\endcsname{\let\PY@bf=\textbf\def\PY@tc##1{\textcolor[rgb]{0.73,0.40,0.13}{##1}}}
\expandafter\def\csname PY@tok@sr\endcsname{\def\PY@tc##1{\textcolor[rgb]{0.73,0.40,0.53}{##1}}}
\expandafter\def\csname PY@tok@ss\endcsname{\def\PY@tc##1{\textcolor[rgb]{0.10,0.09,0.49}{##1}}}
\expandafter\def\csname PY@tok@sx\endcsname{\def\PY@tc##1{\textcolor[rgb]{0.00,0.50,0.00}{##1}}}
\expandafter\def\csname PY@tok@m\endcsname{\def\PY@tc##1{\textcolor[rgb]{0.40,0.40,0.40}{##1}}}
\expandafter\def\csname PY@tok@gh\endcsname{\let\PY@bf=\textbf\def\PY@tc##1{\textcolor[rgb]{0.00,0.00,0.50}{##1}}}
\expandafter\def\csname PY@tok@gu\endcsname{\let\PY@bf=\textbf\def\PY@tc##1{\textcolor[rgb]{0.50,0.00,0.50}{##1}}}
\expandafter\def\csname PY@tok@gd\endcsname{\def\PY@tc##1{\textcolor[rgb]{0.63,0.00,0.00}{##1}}}
\expandafter\def\csname PY@tok@gi\endcsname{\def\PY@tc##1{\textcolor[rgb]{0.00,0.63,0.00}{##1}}}
\expandafter\def\csname PY@tok@gr\endcsname{\def\PY@tc##1{\textcolor[rgb]{1.00,0.00,0.00}{##1}}}
\expandafter\def\csname PY@tok@ge\endcsname{\let\PY@it=\textit}
\expandafter\def\csname PY@tok@gs\endcsname{\let\PY@bf=\textbf}
\expandafter\def\csname PY@tok@gp\endcsname{\let\PY@bf=\textbf\def\PY@tc##1{\textcolor[rgb]{0.00,0.00,0.50}{##1}}}
\expandafter\def\csname PY@tok@go\endcsname{\def\PY@tc##1{\textcolor[rgb]{0.53,0.53,0.53}{##1}}}
\expandafter\def\csname PY@tok@gt\endcsname{\def\PY@tc##1{\textcolor[rgb]{0.00,0.27,0.87}{##1}}}
\expandafter\def\csname PY@tok@err\endcsname{\def\PY@bc##1{\setlength{\fboxsep}{0pt}\fcolorbox[rgb]{1.00,0.00,0.00}{1,1,1}{\strut ##1}}}
\expandafter\def\csname PY@tok@kc\endcsname{\let\PY@bf=\textbf\def\PY@tc##1{\textcolor[rgb]{0.00,0.50,0.00}{##1}}}
\expandafter\def\csname PY@tok@kd\endcsname{\let\PY@bf=\textbf\def\PY@tc##1{\textcolor[rgb]{0.00,0.50,0.00}{##1}}}
\expandafter\def\csname PY@tok@kn\endcsname{\let\PY@bf=\textbf\def\PY@tc##1{\textcolor[rgb]{0.00,0.50,0.00}{##1}}}
\expandafter\def\csname PY@tok@kr\endcsname{\let\PY@bf=\textbf\def\PY@tc##1{\textcolor[rgb]{0.00,0.50,0.00}{##1}}}
\expandafter\def\csname PY@tok@bp\endcsname{\def\PY@tc##1{\textcolor[rgb]{0.00,0.50,0.00}{##1}}}
\expandafter\def\csname PY@tok@fm\endcsname{\def\PY@tc##1{\textcolor[rgb]{0.00,0.00,1.00}{##1}}}
\expandafter\def\csname PY@tok@vc\endcsname{\def\PY@tc##1{\textcolor[rgb]{0.10,0.09,0.49}{##1}}}
\expandafter\def\csname PY@tok@vg\endcsname{\def\PY@tc##1{\textcolor[rgb]{0.10,0.09,0.49}{##1}}}
\expandafter\def\csname PY@tok@vi\endcsname{\def\PY@tc##1{\textcolor[rgb]{0.10,0.09,0.49}{##1}}}
\expandafter\def\csname PY@tok@vm\endcsname{\def\PY@tc##1{\textcolor[rgb]{0.10,0.09,0.49}{##1}}}
\expandafter\def\csname PY@tok@sa\endcsname{\def\PY@tc##1{\textcolor[rgb]{0.73,0.13,0.13}{##1}}}
\expandafter\def\csname PY@tok@sb\endcsname{\def\PY@tc##1{\textcolor[rgb]{0.73,0.13,0.13}{##1}}}
\expandafter\def\csname PY@tok@sc\endcsname{\def\PY@tc##1{\textcolor[rgb]{0.73,0.13,0.13}{##1}}}
\expandafter\def\csname PY@tok@dl\endcsname{\def\PY@tc##1{\textcolor[rgb]{0.73,0.13,0.13}{##1}}}
\expandafter\def\csname PY@tok@s2\endcsname{\def\PY@tc##1{\textcolor[rgb]{0.73,0.13,0.13}{##1}}}
\expandafter\def\csname PY@tok@sh\endcsname{\def\PY@tc##1{\textcolor[rgb]{0.73,0.13,0.13}{##1}}}
\expandafter\def\csname PY@tok@s1\endcsname{\def\PY@tc##1{\textcolor[rgb]{0.73,0.13,0.13}{##1}}}
\expandafter\def\csname PY@tok@mb\endcsname{\def\PY@tc##1{\textcolor[rgb]{0.40,0.40,0.40}{##1}}}
\expandafter\def\csname PY@tok@mf\endcsname{\def\PY@tc##1{\textcolor[rgb]{0.40,0.40,0.40}{##1}}}
\expandafter\def\csname PY@tok@mh\endcsname{\def\PY@tc##1{\textcolor[rgb]{0.40,0.40,0.40}{##1}}}
\expandafter\def\csname PY@tok@mi\endcsname{\def\PY@tc##1{\textcolor[rgb]{0.40,0.40,0.40}{##1}}}
\expandafter\def\csname PY@tok@il\endcsname{\def\PY@tc##1{\textcolor[rgb]{0.40,0.40,0.40}{##1}}}
\expandafter\def\csname PY@tok@mo\endcsname{\def\PY@tc##1{\textcolor[rgb]{0.40,0.40,0.40}{##1}}}
\expandafter\def\csname PY@tok@ch\endcsname{\let\PY@it=\textit\def\PY@tc##1{\textcolor[rgb]{0.25,0.50,0.50}{##1}}}
\expandafter\def\csname PY@tok@cm\endcsname{\let\PY@it=\textit\def\PY@tc##1{\textcolor[rgb]{0.25,0.50,0.50}{##1}}}
\expandafter\def\csname PY@tok@cpf\endcsname{\let\PY@it=\textit\def\PY@tc##1{\textcolor[rgb]{0.25,0.50,0.50}{##1}}}
\expandafter\def\csname PY@tok@c1\endcsname{\let\PY@it=\textit\def\PY@tc##1{\textcolor[rgb]{0.25,0.50,0.50}{##1}}}
\expandafter\def\csname PY@tok@cs\endcsname{\let\PY@it=\textit\def\PY@tc##1{\textcolor[rgb]{0.25,0.50,0.50}{##1}}}

\def\PYZbs{\char`\\}
\def\PYZus{\char`\_}
\def\PYZob{\char`\{}
\def\PYZcb{\char`\}}
\def\PYZca{\char`\^}
\def\PYZam{\char`\&}
\def\PYZlt{\char`\<}
\def\PYZgt{\char`\>}
\def\PYZsh{\char`\#}
\def\PYZpc{\char`\%}
\def\PYZdl{\char`\$}
\def\PYZhy{\char`\-}
\def\PYZsq{\char`\'}
\def\PYZdq{\char`\"}
\def\PYZti{\char`\~}
% for compatibility with earlier versions
\def\PYZat{@}
\def\PYZlb{[}
\def\PYZrb{]}
\makeatother


    % Exact colors from NB
    \definecolor{incolor}{rgb}{0.0, 0.0, 0.5}
    \definecolor{outcolor}{rgb}{0.545, 0.0, 0.0}



    
    % Prevent overflowing lines due to hard-to-break entities
    \sloppy 
    % Setup hyperref package
    \hypersetup{
      breaklinks=true,  % so long urls are correctly broken across lines
      colorlinks=true,
      urlcolor=urlcolor,
      linkcolor=linkcolor,
      citecolor=citecolor,
      }
    % Slightly bigger margins than the latex defaults
    
    \geometry{verbose,tmargin=1in,bmargin=1in,lmargin=1in,rmargin=1in}
    
    

    \begin{document}
    
    
    \maketitle
    
    

    
    \section{18ma573v01}\label{ma573v01}

\subsection{COMPUTATIONAL METHODS OF FINANCIAL
MATHEMATICS}\label{computational-methods-of-financial-mathematics}

\textbf{General course info}

\begin{itemize}
\tightlist
\item
  Syllabus - \href{doc/syllabus_v01.pdf}{pdf}
\item
  Final and capston project -\href{./doc/capstone.pdf}{pdf}
\item
  Mapping - \href{src/mapping.ipynb}{ipynb}
\end{itemize}

    \textbf{Python basics on Jupyter notebook}

In this section, we will get familiar with python language with Jupyter
notebook through some financial applications.

\begin{itemize}
\tightlist
\item
  Environment setup
  \href{https://github.com/songqsh/18ma573pub/blob/master/src/first_notebook_v01.ipynb}{ipynb}
\item
  Python on Jupyter notebook \href{src/python_notebook.ipynb}{ipynb}
\item
  Pandas \href{src/pandas_basics.ipynb}{ipynb}
\end{itemize}

\textbf{Finite difference operators with python functions}

Below, we will discuss some simple coding related to finite difference
operators and convergence rate. This will help you to get to know basic
coding with python functions. However, it's your responsibility to get
to know enough coding via online resource.

\begin{itemize}
\tightlist
\item
  First order finite difference operators -
  \href{src/first_fd_v01.ipynb}{ipynb}
\item
  Convergence order - \href{src/ffd_convergence_rate_v01.ipynb}{ipynb}

  \begin{itemize}
  \tightlist
  \item
    (hw) Second order finite difference operator -
    \href{src/second_fd.ipynb}{ipynb}

    \begin{itemize}
    \tightlist
    \item
      soln - \href{src/second_fd_soln.ipynb}{ipynb}
    \end{itemize}
  \item
    (hw) Finite difference operator with higher order convergence -
    \href{src/ex_fd.ipynb}{ipynb}

    \begin{itemize}
    \tightlist
    \item
      soln - \href{src/ex_fd_soln.ipynb}{ipynb}
    \end{itemize}
  \item
    (ex) FD - \href{doc/fd_ex.pdf}{ipynb}
  \end{itemize}
\end{itemize}

\textbf{BSM option price}

\begin{itemize}
\tightlist
\item
  European call/put option class -
  \href{https://github.com/songqsh/18ma573pub/blob/master/src/european_options_class.ipynb}{ipynb}

  \begin{itemize}
  \tightlist
  \item
    (hw) payoff diagram of butterfly -
    \href{https://github.com/songqsh/18ma573pub/blob/master/src/option_combinations.ipynb}{ipynb}
  \end{itemize}
\item
  BSM formula -
  \href{https://github.com/songqsh/18ma573pub/blob/master/src/bsm_formula_v01.ipynb}{ipynb}

  \begin{itemize}
  \tightlist
  \item
    (hw) Bsm price change -
    \href{https://github.com/songqsh/18ma573pub/blob/master/src/bsm_price_change.ipynb}{ipynb}
  \end{itemize}
\item
  Importing modules -
  \href{https://github.com/songqsh/18ma573pub/blob/master/src/import_modules.md}{md}
\end{itemize}

    \textbf{Calibration}

\begin{itemize}
\tightlist
\item
  implied volatility -
  \href{https://nbviewer.jupyter.org/github/songqsh/18ma573pub/blob/master/src/implied_vol_v01.ipynb}{ipynb}

  \begin{itemize}
  \tightlist
  \item
    (hw) IV -
    \href{https://nbviewer.jupyter.org/github/songqsh/18ma573pub/blob/master/src/hw_implied_vol.ipynb}{ipynb}
  \item
    (ex) Sensitivity - \href{src/ex_montonicity.ipynb}{ipynb}
  \end{itemize}
\item
  volatility smile - \href{src/vol_smile_v01.ipynb}{ipynb}
\item
  bsm\_calibration - \href{src/bsm_calibration_v01.ipynb}{ipynb}

  \begin{itemize}
  \tightlist
  \item
    (hw) -
    \href{https://nbviewer.jupyter.org/github/songqsh/18ma573pub/blob/master/src/hw_bsm_calibration.ipynb}{ipynb}
  \end{itemize}
\item
  BSM geometric asian option -
  \href{src/bsm_geometric_asian_option.ipynb}{ipynb}

  \begin{itemize}
  \tightlist
  \item
    (hw) calibration performance -
    \href{src/hw_bsm_geometric_asian_option.ipynb}{ipynb}

    \begin{itemize}
    \tightlist
    \item
      (soln) \href{src/calibration_performance_v01.ipynb}{ipynb}
    \end{itemize}
  \end{itemize}
\end{itemize}

    \textbf{Monte Carlo basics}

\begin{itemize}
\tightlist
\item
  Monte Carlo basics: Estimating \(\pi\) -
  \href{./doc/pi_mc_01.pdf}{pdf} -
  \href{https://github.com/songqsh/18ma573pub/blob/master/src/pi.ipynb}{ipynb}

  \begin{itemize}
  \tightlist
  \item
    (hw) \href{doc/hw_mc_01.pdf}{pdf}
  \item
    (hw)
    \href{https://github.com/songqsh/18ma573pub/blob/master/src/hw_mc_02.ipynb}{ipynb}
  \item
    (ex) \href{src/ex_omc_01.ipynb}{ipynb}
  \end{itemize}
\item
  Ordinary Monte Carlo: Definite integral
  \href{https://github.com/songqsh/18ma573pub/blob/master/doc/omc_integral_01.pdf}{pdf}

  \begin{itemize}
  \tightlist
  \item
    (hw) \href{doc/hw_omc_integral.pdf}{pdf}
  \end{itemize}
\item
  Inverse transform method + Importance sampling: Definite integral -
  \href{./doc/is_it_integral.pdf}{pdf}

  \begin{itemize}
  \tightlist
  \item
    (hw) - \href{./doc/hw_is_it_integral.pdf}{pdf}
  \end{itemize}
\item
  Exact sampling of a Brownian path - \href{doc/bm_1d.pdf}{pdf}

  \begin{itemize}
  \tightlist
  \item
    (hw) bsm + aac + exact sampling -
    \href{doc/hw_exact_sample.pdf}{pdf}

    \begin{itemize}
    \tightlist
    \item
      (soln) \href{src/soln_exact_sample.ipynb}{ipynb}
    \end{itemize}
  \item
    (hw) correlations to aac - \href{doc/hw_payoff_correlation.pdf}{pdf}

    \begin{itemize}
    \tightlist
    \item
      (soln) \href{src/soln_payoff_correlation.ipynb}{ipynb}
    \end{itemize}
  \item
    (hw) vasicek model calibration -
    \href{src/hw_vasicek_calibration.ipynb}{ipynb}

    \begin{itemize}
    \tightlist
    \item
      (soln part 1) -
      \href{src/hw_vasicek_calibration_soln_part1.ipynb}{ipynb}
    \item
      (soln part 2) -
      \href{src/calibration_vasicek_libor_swap_part2.ipynb}{ipynb}
    \end{itemize}
  \end{itemize}
\end{itemize}

    \textbf{Fourier method} - References: -
\href{other/paper/ChiXX_Fourier.pdf}{pdf by Chiu} -
\href{other/paper/CM99_FFT.pdf}{pdf by Carr and Madan 1999} - Fourier
method - \href{src/fourier_method.ipynb}{ipynb}\\
- demo - \href{src/fourier.ipynb}{ipynb} - Fourier method by Carr-Madan
- \href{src/fourier_carr_madan.ipynb}{ipynb}

    \textbf{Euler scheme} - 1d Euler scheme -
\href{./doc/euler_sde_1d.pdf}{pdf} - Demo: bm\_1d\_path -
\href{./src/bm_1d_path.ipynb}{ipynb} - Demo: euler\_1d\_path -
\href{./src/euler_1d.ipynb}{ipynb} - 2d Euler scheme -
\href{./doc/euler_sde_2d.pdf}{pdf} - Call on Heston model -
\href{./src/euler_heston.ipynb}{ipynb} - Heston European option data
cook - \href{./src/heston_data_cook.ipynb}{ipynb} - geometric asian
option + Heston + Euler -
\href{src/heston_geometric_asian_euler.ipynb}{ipynb}

    \textbf{Finite difference method}

\begin{itemize}
\tightlist
\item
  FTCS with a toy - \href{src/ftcs_stability_heat_toy.ipynb}{ipynb} and
  \href{./doc/stability_ftcs_01.pdf}{pdf}

  \begin{itemize}
  \tightlist
  \item
    (hw) FTCS and stability: Heat equation with Cauchy-Dirichlet data -
    \href{./src/ftcs_stability_heat_1d.ipynb}{ipynb}
  \item
    (hw) FTCS stability questions -
    \href{src/hw_ftcs_stability.ipynb}{ipynb}
  \end{itemize}
\item
  BTCS and CRR model - \href{doc/fdm_crr.pdf}{pdf}

  \begin{itemize}
  \tightlist
  \item
    (hw) \href{doc/hw_crr.pdf}{pdf}

    \begin{itemize}
    \tightlist
    \item
      (soln by) \href{src/L05s01.ipynb}{song}
    \end{itemize}
  \item
    BTCS and stability: Heat equation -
    \href{./src/btcs_stability_heat_1d.ipynb}{ipynb}
  \end{itemize}
\item
  Stability analysis of FTCS - \href{./doc/stability_ftcs_02.pdf}{pdf}
\item
  Crank-Nicolson scheme - \href{./doc/stability_ftcs_03.pdf}{pdf}
\end{itemize}

    \section{Explicit pricing formula}\label{explicit-pricing-formula}

\begin{itemize}
\tightlist
\item
  \href{./src/explicit_bsm_eu.ipynb}{Explicit Option Price: BSM+Eu}
\item
  \href{./src/explicit_bsm_greeks.ipynb}{Project: BSM Greeks}

  \begin{itemize}
  \tightlist
  \item
    \href{./src/explicit_bsm_greeks_soln.ipynb}{Soln}
  \end{itemize}
\item
  \href{./src/explicit_vasicek_zcb.ipynb}{Explicit Zero Coupon price:
  Vasicek+Zcb}
\item
  \href{./src/calibration_vasicek_libor_swap.ipynb}{Calibration: Vasicek
  + Libor + Swap}
\end{itemize}

\section{Monte Carlo - Exact
sampling}\label{monte-carlo---exact-sampling}

\begin{itemize}
\tightlist
\item
  \href{./src/es_bsm_eu.ipynb}{Exact Sampling: BSM+Eu}
\item
  \href{./src/es_bsm_knock_in.ipynb}{Exact sampling: BSM + Knock-in}
\item
  \href{./src/is_bsm_knock_in.ipynb}{Importance sampling: BSM +
  Knock-in}
\item
  \href{./src/control_variates_bsm_aac.ipynb}{Control Variates: BSM+Aac}
\end{itemize}

\section{Problem Sets}\label{problem-sets}

\begin{itemize}
\tightlist
\item
  \href{./src/problems01.ipynb}{Problems01}
\end{itemize}


    % Add a bibliography block to the postdoc
    
    
    
    \end{document}
