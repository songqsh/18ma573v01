\documentclass{article}
\usepackage[pagebackref,letterpaper=true,colorlinks=true,pdfpagemode=none,urlcolor=blue,linkcolor=blue,citecolor=blue,pdfstartview=FitH]{hyperref}

\usepackage{amsmath,amsfonts}
\usepackage{graphicx}
\usepackage{color}


\usepackage{algorithm}
\usepackage[noend]{algpseudocode}

\setlength{\oddsidemargin}{0pt}
\setlength{\evensidemargin}{0pt}
\setlength{\textwidth}{6.0in}
\setlength{\topmargin}{0in}
\setlength{\textheight}{8.5in}


\setlength{\parindent}{0in}
\setlength{\parskip}{5px}

%\input{macrosblog}

%%%%%%%%% For wordpress conversion

\def\more{}

\newif\ifblog
\newif\iftex
\blogfalse
\textrue


\usepackage{ulem}
\def\em{\it}
\def\emph#1{\textit{#1}}

\def\image#1#2#3{\begin{center}\includegraphics[#1pt]{#3}\end{center}}

\let\hrefnosnap=\href

\newenvironment{btabular}[1]{\begin{tabular} {#1}}{\end{tabular}}

\newenvironment{red}{\color{red}}{}
\newenvironment{green}{\color{green}}{}
\newenvironment{blue}{\color{blue}}{}

%%%%%%%%% Typesetting shortcuts

\def\B{\{0,1\}}
\def\xor{\oplus}

\def\P{{\mathbb P}}
\def\E{{\mathbb E}}
\def\var{{\bf Var}}

\def\N{{\mathbb N}}
\def\Z{{\mathbb Z}}
\def\R{{\mathbb R}}
\def\C{{\mathbb C}}
\def\Q{{\mathbb Q}}
\def\eps{{\epsilon}}

\def\bz{{\bf z}}

\def\true{{\tt true}}
\def\false{{\tt false}}

%%%%%%%%% Theorems and proofs

\newtheorem{exercise}{Exercise}
\newtheorem{theorem}{Theorem}
\newtheorem{lemma}[theorem]{Lemma}
\newtheorem{definition}[theorem]{Definition}
\newtheorem{corollary}[theorem]{Corollary}
\newtheorem{proposition}[theorem]{Proposition}
\newtheorem{example}{Example}
\newtheorem{remark}[theorem]{Remark}
\newenvironment{proof}{\noindent {\sc Proof:}}{$\Box$} %\medskip} 
%%%%%%%%% I added
\newtheorem{assumption}{Assumption}
%%%%%%%%

\begin{document}

\section{Abstract}
You will learn Exact sampling of Brownian path and Geometric Brownian path

\begin{itemize}
 \item Exact sampling of Brownian path
 \item Exact sampling of Geometric Brownian path
\end{itemize}

Reference: 

[1] Section 3.1 of [Gla03]: Random walk construction

\section{Anaysis}
\subsection{Brownian path}
Let time mesh $\Pi$ be of the form
$$\Pi = \{0 = t_0 \le t_1 \le \ldots \le t_N = T\}.$$
We use
$$\langle W, \Pi\rangle  = \{W(t): t\in \Pi\}$$
the projection of the brownian path on $\Pi$.
To have a simulation of Brownian path by random walk, one can iterate (3.2) of [1], i.e.
\begin{equation}\label{eq:01}
W(t_{i+1}) = W(t_i) + \sqrt{t_{i+1} - t_i} Z_{i+1}.
\end{equation}

\begin{example}
\label{exm:01}
Let uniform mesh be denoted by $$\Pi_{T, N} = \{i T/N: i = 0, ..., N\}.$$
\begin{itemize}
\item 
Write pseudocode.
\begin{algorithm}
\caption{Use  \eqref{eq:01}, generate $\hat W$ to simulate a discrete path $\langle W, \Pi_{T, N}\rangle$. }
\label{alg:bm}
\begin{algorithmic}[1]
\Procedure{exactbm1d}{$T, N$}\Comment{$T, N$ is ...}
\State   ...
\State  ...
\EndProcedure
\end{algorithmic}
\end{algorithm}
\item Prove that $\hat W$ is an exact sampling.
\item Draw 10 path simulations of $t\mapsto \frac{W(t)}{\sqrt{2t \log\log t}}$ on interval $t= [100, 110]$ with mesh size $h = 0.1$.
\end{itemize}
\end{example}

\subsection{Geometric Brownian path}
$GBM(x_0, r, \sigma)$  is given by
$$X(t) = x_0 \exp\{(r - \frac 1 2 \sigma^2)t + \sigma W(t)\}.$$
We can replace $W(t)$ by its exact simulation $\hat W(t)$ to get exact simulation of $X(t)$, i.e.
\begin{equation}\label{eq:gbmhat}
\hat X(t) = x_0 \exp\{(r - \frac 1 2 \sigma^2)t + \sigma \hat W(t)\}.
\end{equation}


\begin{example}\label{exm:gbm}
Let $\Pi_{T, N}$ be the uniform mesh and $X$ be $GBM(x_0, r, \sigma)$.
\begin{itemize}
\item 
Write pseudocode.
\begin{algorithm}
\caption{ Use  \eqref{eq:gbmhat}, generate $\hat X$ to simulate a discrete path $\langle X, \Pi_{T, N}\rangle$.  }
\label{alg:gbm}
\begin{algorithmic}[1]
\Procedure{exactgbm1d}{$T, N$}\Comment{$T, N$ is ...}
\State   ...
\State  ...
\EndProcedure
\end{algorithmic}
\end{algorithm}

\end{itemize}


\end{example}

\subsection{Application to Arithmetic asian option price}
Arithmetic asian call option with maturity $T$ and strick $K$ has its pay off as
$$C(T) = (A(T) - K)^+,$$
where $A(T)$ is arithmetic average of the stock price at times
$0 \le t_1 < t_2, \ldots, < t_n = T$, i.e.
$$A(T) = \frac{1}{n} \sum_{i=1}^n S(t_i).$$

The call price can be thus written by
$$C_0 = \mathbb E [e^{-rT} (A(T) - K)^+].$$


Unlike the geometric asian option, arithmetic counterpart does not have explicit formula for its price.
In this below, we shall use MC. 
In practice, an arithmetic asian option with a given number $n$ 
of time steps takes the price average at $n+1$ points
$$t_i = (i-1) \frac{T}{n}, \quad i = 1, 2, \ldots, (n+1).$$

\begin{example}\label{exm:aao}  Consider  Arithmetic asian option price on BSM by exact sampling.
\begin{itemize}
\item Write a pseudocode for Arithmetic asian option price on BSM
\item To the Gbm class, add a method 
\begin{center} 
{\rm arasian(otype, strike, maturity, nstep, npath)}
\end{center} 
for the price by exact sampling.
\item Use your code to compute Arithmetic asian option of 
$$S_0 = 100.0, \sigma= 0.20, r=0.0475, K = 110.0, T = 1.0, otype = 1, nstep = 5.$$
\end{itemize}
\end{example}


\end{document}
