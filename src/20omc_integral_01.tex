\documentclass{article}
\usepackage[pagebackref,letterpaper=true,colorlinks=true,pdfpagemode=none,urlcolor=blue,linkcolor=blue,citecolor=blue,pdfstartview=FitH]{hyperref}

\usepackage{amsmath,amsfonts}
\usepackage{graphicx}
\usepackage{color}


\usepackage{algorithm}
\usepackage[noend]{algpseudocode}

\setlength{\oddsidemargin}{0pt}
\setlength{\evensidemargin}{0pt}
\setlength{\textwidth}{6.0in}
\setlength{\topmargin}{0in}
\setlength{\textheight}{8.5in}


\setlength{\parindent}{0in}
\setlength{\parskip}{5px}

%\input{macrosblog}

%%%%%%%%% For wordpress conversion

\def\more{}

\newif\ifblog
\newif\iftex
\blogfalse
\textrue


\usepackage{ulem}
\def\em{\it}
\def\emph#1{\textit{#1}}

\def\image#1#2#3{\begin{center}\includegraphics[#1pt]{#3}\end{center}}

\let\hrefnosnap=\href

\newenvironment{btabular}[1]{\begin{tabular} {#1}}{\end{tabular}}

\newenvironment{red}{\color{red}}{}
\newenvironment{green}{\color{green}}{}
\newenvironment{blue}{\color{blue}}{}

%%%%%%%%% Typesetting shortcuts

\def\B{\{0,1\}}
\def\xor{\oplus}

\def\P{{\mathbb P}}
\def\E{{\mathbb E}}
\def\var{{\bf Var}}

\def\N{{\mathbb N}}
\def\Z{{\mathbb Z}}
\def\R{{\mathbb R}}
\def\C{{\mathbb C}}
\def\Q{{\mathbb Q}}
\def\eps{{\epsilon}}

\def\bz{{\bf z}}

\def\true{{\tt true}}
\def\false{{\tt false}}

%%%%%%%%% Theorems and proofs

\newtheorem{exercise}{Exercise}
\newtheorem{theorem}{Theorem}
\newtheorem{lemma}[theorem]{Lemma}
\newtheorem{definition}[theorem]{Definition}
\newtheorem{corollary}[theorem]{Corollary}
\newtheorem{proposition}[theorem]{Proposition}
\newtheorem{example}{Example}
\newtheorem{remark}[theorem]{Remark}
\newenvironment{proof}{\noindent {\sc Proof:}}{$\Box$} %\medskip} 
%%%%%%%%% I added
\newtheorem{assumption}{Assumption}
%%%%%%%%

\begin{document}

\section{Abstract}
You will learn
\begin{itemize}
 \item taking definite integral by ordinary Monte Carlo (OMC)
 \item exact sampling with python provided random number generators
\end{itemize}

\section{Problem}

\begin{example}\label{ex:integral}
Our goal is to compute, using OMC by exact sampling
$$\alpha = \int_0^1 h(x) dx$$
where
$$h(x) = 100 \cdot I_{(0, 1/100]}(x) + 1\cdot I_{(1/100, 1)} (x).$$
The exact value shall be 
$$\alpha = 1.99.$$
\end{example}

\section{Analysis}

\subsection{OMC by exact sampling}
The objective is to 
\begin{itemize}
 \item estimate 
$$\alpha = \mathbb E[X], \quad X \sim p(x)$$
one can use random number generator by computer (if possible)
$$\{iid \ X_i \sim p(x): i = 1, 2, \ldots, n, \} .$$
\end{itemize}

Then, one can compute the approximation of $\alpha$ by
$$\hat \alpha_n = \frac 1 n \sum_{i=1}^n X_i.$$

We say $\hat \alpha_n$ as OMC by {\it exact sampling}, since the sample $X_i$
produced by random generator has the same distribution as true distribution $X$, 
i.e. 
$$X_i \sim X, \ \forall i.$$

 

\begin{example}
Prove the  properties of the OMC by exact sampling below:
\begin{itemize}
 \item $X_1$ itself can be treated as an unbiased MC, because
 $$\mathbb E [X_1] = \alpha.$$
 However, MSE is big, ie.
 $$MSE(X_1) = Var(X) = \int x^2 p(x) dx.$$
 \item $\hat \alpha_n$ is {\it consistent almost surely} due to LLN, i.e.
 $$\hat \alpha_n \to \alpha, \hbox{ almost surely as } n\to \infty.$$
 Moreover, $\hat \alpha_n$ is unbiased too, and
 $$MSE(\hat \alpha_n) = Var(\hat \alpha_n) = \frac 1 n Var(X) \to 0.$$
\end{itemize}
\end{example}

\subsection{Evaluation of integral}
Back to our Example \ref{ex:integral}, we write
$$\alpha = \mathbb E[X] = \mathbb E[h(Y)],$$
where $X = h(Y)$ and $Y\sim U(0,1)$.
In other words, although $X$-sampling is not directly available in python, 
one can use $U(0,1)$ random generator (see {\it numpy.random.uniform}) to produce $Y_i$, then compute $h(Y_i)$ for 
the sample $X_i$.


\begin{algorithm}
\caption{Integral by MC - Example \ref{ex:integral}}
\label{alg:mcpi}
\begin{algorithmic}[1]
\Procedure{mcintegral}{$N$}\Comment{$N$ is total number of samples}
\State  $s \gets 0$ \Comment{$s$ is the sum of samples}
\For {$i = 1...N$}
	\State generate two numbers $Y$ from $U(0,1)$ \Comment{use  {\it numpy.random.uniform}}
	\State $s\gets s+h(Y)$ 
\EndFor
\State \textbf{return} $\frac{s}{N}$ \Comment{return the average}
\EndProcedure
\end{algorithmic}
\end{algorithm}




\end{document}
