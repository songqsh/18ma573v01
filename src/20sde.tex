\documentclass{article}
\usepackage[pagebackref,letterpaper=true,colorlinks=true,pdfpagemode=none,urlcolor=blue,linkcolor=blue,citecolor=blue,pdfstartview=FitH]{hyperref}

\usepackage{amsmath,amsfonts}
\usepackage{graphicx}
\usepackage{color}


\usepackage{algorithm}
\usepackage[noend]{algpseudocode}


\setlength{\oddsidemargin}{0pt}
\setlength{\evensidemargin}{0pt}
\setlength{\textwidth}{6.0in}
\setlength{\topmargin}{0in}
\setlength{\textheight}{8.5in}


\setlength{\parindent}{0in}
\setlength{\parskip}{5px}

%\input{macrosblog}

%%%%%%%%% For wordpress conversion

\def\more{}

\newif\ifblog
\newif\iftex
\blogfalse
\textrue


\usepackage{ulem}
\def\em{\it}
\def\emph#1{\textit{#1}}

\def\image#1#2#3{\begin{center}\includegraphics[#1pt]{#3}\end{center}}

\let\hrefnosnap=\href

\newenvironment{btabular}[1]{\begin{tabular} {#1}}{\end{tabular}}

\newenvironment{red}{\color{red}}{}
\newenvironment{green}{\color{green}}{}
\newenvironment{blue}{\color{blue}}{}

%%%%%%%%% Typesetting shortcuts

\def\B{\{0,1\}}
\def\xor{\oplus}

\def\P{{\mathbb P}}
\def\E{{\mathbb E}}
\def\var{{\bf Var}}

\def\N{{\mathbb N}}
\def\Z{{\mathbb Z}}
\def\R{{\mathbb R}}
\def\C{{\mathbb C}}
\def\Q{{\mathbb Q}}
\def\eps{{\epsilon}}

\def\bz{{\bf z}}

\def\true{{\tt true}}
\def\false{{\tt false}}

%%%%%%%%% Theorems and proofs

\newtheorem{exercise}{Exercise}
\newtheorem{theorem}{Theorem}
\newtheorem{lemma}[theorem]{Lemma}
\newtheorem{definition}[theorem]{Definition}
\newtheorem{corollary}[theorem]{Corollary}
\newtheorem{proposition}[theorem]{Proposition}
\newtheorem{example}{Example}
\newtheorem{remark}[theorem]{Remark}
\newenvironment{proof}{\noindent {\sc Proof:}}{$\Box$} %\medskip} 
%%%%%%%%% I added
\newtheorem{assumption}{Assumption}
%%%%%%%%

\begin{document}

\section{Abstract}
\begin{itemize}
 \item SDE 
 \item and related financial models
\end{itemize}


\section{SDE}
\subsection{General problem}
We will consider the general d-dimensional SDE:
$$d X_{t} = b(X_{t}) dt + \sigma(X_{t}) dW_{t}, X_{0} = x_{0}$$
where $b: \mathbb R^{d} \mapsto \mathbb R^{d}$ is a smooth 
vector field  on  $\mathbb R^{d}$,
$\sigma: \mathbb R^{d} \mapsto \mathbb R^{d\times d}$ is a smooth 
matrix-valued function, $W$ is a d-dimensional standard Brownian motion, 
and $x_{0}$ is the initial d-dimensional vector. 


Some theoretical interests are the sufficient condition for the unique solvability, and computations, which can be founded in the literature.


\subsection{Example: 2-d SDE}
It can be written by system of two 1-d SDEs as the following:
$$
\left\{
\begin{array}
 {ll}
 d X_{1,t} = b_{1,t} dt + \sigma_{11,t}dW_{1,t} + \sigma_{12,t} dW_{2,t}, 
 & X_{1,0} = x_{1,0}\\
 d X_{2,t} = b_{2,t} dt + \sigma_{21,t}dW_{1,t} + \sigma_{22,t} dW_{2,t}, 
 & X_{2,0} = x_{2,0}
\end{array}
\right.
$$
In the above, we assume $W_{1}$ and $W_{2}$ are two independent 1-d Brownian motions.


\section{Stock models}

\subsection{Arithmetic BM}
We denote by $BM(\mu, \sigma^2)$ the dynamics
$$d X_t = \mu dt + \sigma dW_t.$$

\subsection{Geometric BM}
We denote by $GBM(s, \mu, \sigma^2)$ the dynamics
$$d S_t = \mu S_t dt + \sigma S_t dW_t, S_{0} = s$$
Non-negativity of the GBM process is good for modeling stock price, namely BSM.
\begin{example}
\label{exm:gbm01}
Find $\log S_t$ for $S \sim GBM(s, \mu, \sigma^2)$.
\end{example}

\subsection{Stochastic volatility model: Local volatility}
Due to limit capacity of GBM in calibration, one can extend the asset price as
$$d S_t = \mu S_t dt + \sigma_t S_t dW_t.$$
The difference is that the volatility $\sigma_t$ is a random process and this model is classified as stochastic volatility model.

If volatility is modelled by $\sigma_t = \hat \sigma(t, S_t)$ for some deterministic function $\hat \sigma$, then it is called local volatility model, one of the most important case in stochastic volatility models.
\subsubsection{CEV}
The stock follows
$$d S_{t} = \mu S_{t} dt + \sigma S_{t}^{\gamma} dW_{t}.$$
\begin{itemize}
\item $\gamma = 1$ gives GBM.
 \item  When $\gamma <1$,  we see the so-called leverage effect, commonly observed in equity markets, where the volatility of a stock increases as its price falls. 
 \item Conversely, when $\gamma >1$, it exhibits so-called inverse leverage effect often observed in commodity markets, whereby the volatility of the price of a commodity tends to increase as its price increases.
\end{itemize}

\subsection{Stochastic volatility model: Heston model}
Heston model as a stochastic volatility model belongs to 2-d SDE in the above. However, the domain of the diffusion matrix $\sigma$ is not entire 2-d space. 

In the Heston model, the dynamic involves two processes $(S_{t}, \nu_{t})$.
More precisely, the asset price $S$ follows generalized geometric Brownian motion with random volatility process $\sqrt{\nu_{t}}$, i.e.  
$$d S_{t} = r S_{t} dt + \sqrt{\nu_{t}} S_{t} dW_{1,t},$$
while squared of volatility process $\nu$ follows CIR process
$$ d \nu_{t} = \kappa (\theta - \nu_{t}) dt + \xi \sqrt{\nu_{t}} (\rho dW_{1,t} + 
\bar \rho d W_{2,t})$$
with $\rho^{2} + \bar \rho^{2} = 1.$ 

\begin{itemize}
\item
Feller condition for its existence of the solution is
$$2\kappa \theta > \xi^{2}.$$
\end{itemize}


\begin{example}
\label{exm:heston01}
Our goal is to adapt the above Euler scheme to Heston model with the following parameters:
$$ S_{0} = 100, \nu(0) = .04, r = .05, \kappa = 1.2, 
\theta = .04, \xi = .3, \rho = .5.$$
The estimation of Call$(T =1, K = 100)$ is given as 10.3009, see Page 357 of \cite{Gla04}. We will use this for our comparison to our computation.
\end{example}



\section{Short rate models}

In general, interest rate $r_{t}$ is random and 
the zero bond price $P(0, T)$ follows
$$P(0, T) = \mathbb E[ \exp\{- \int_0^T r(u) du\} ].$$

\subsection{Vasicek model}
It is a model for short rate $r_t$ given by OU process:
$$d r_t = \alpha(b - r_t) dt + \sigma dW_t.$$

\subsection{Ho-Lee model}
It is a short rate model given by
$$d r_t = g(t) dt + \sigma dW_t.$$

\subsection{Hull-White model}
It is short rate model, which extends Vasicek model, given by
$$dr_t = [g(t) + h(t) r_t] dt + \sigma(t) dW_t,$$
where $g, h, \sigma$ are given deterministic functions.

\begin{example}
\label{exm:hw01}\begin{itemize}
\item determine function $g, h, \sigma$ for the Vasicek model;
\item write explicit solution for HW.
\end{itemize}
\end{example}


\subsection{CIR model}
It is short rate of
$$d r_t = \alpha(b - r_t) dt + \sigma \sqrt{r_t} dW_t.$$
Note that, squared volatility in Heston model has the same dynamics.

\subsection{Affine term structure: Multifactor model}



We say that a model of $d$-dim factor variable $X_{t}$  is affine if the zero bond can be written as
$$P(t, T) = \exp\{A(t, T) + B^{T}(t, T) X_{t}\}.$$

\begin{example}
\label{exm:mf01}\
Verify that Vasicek model is one-factor affine model.
\end{example}

Indeed, one can have affine class model in more general settings.

\subsubsection{Guassian Multifactor models}
Let the short rate given by 
$$r_t = \mu + \theta^T X_t$$
where $\theta\in \mathbb R^d$, $\mu \in\mathbb R$,  
and $d$-factor process $X_t$ is given by $d$-dimensional OU process
$$d X_t =BX_t dt + K dW_t.$$
Then, it belongs to affine class, see for explicit $P(t, T)$ in  p107 of \cite{Cai04}.




\bibliographystyle{plain}
%\bibliographystyle{plainnat}
%\bibliographystyle{apalike}
\bibliography{../../refs}
\end{document}
