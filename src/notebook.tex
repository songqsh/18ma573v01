
% Default to the notebook output style

    


% Inherit from the specified cell style.




    
\documentclass[11pt]{article}

    
    
    \usepackage[T1]{fontenc}
    % Nicer default font (+ math font) than Computer Modern for most use cases
    \usepackage{mathpazo}

    % Basic figure setup, for now with no caption control since it's done
    % automatically by Pandoc (which extracts ![](path) syntax from Markdown).
    \usepackage{graphicx}
    % We will generate all images so they have a width \maxwidth. This means
    % that they will get their normal width if they fit onto the page, but
    % are scaled down if they would overflow the margins.
    \makeatletter
    \def\maxwidth{\ifdim\Gin@nat@width>\linewidth\linewidth
    \else\Gin@nat@width\fi}
    \makeatother
    \let\Oldincludegraphics\includegraphics
    % Set max figure width to be 80% of text width, for now hardcoded.
    \renewcommand{\includegraphics}[1]{\Oldincludegraphics[width=.8\maxwidth]{#1}}
    % Ensure that by default, figures have no caption (until we provide a
    % proper Figure object with a Caption API and a way to capture that
    % in the conversion process - todo).
    \usepackage{caption}
    \DeclareCaptionLabelFormat{nolabel}{}
    \captionsetup{labelformat=nolabel}

    \usepackage{adjustbox} % Used to constrain images to a maximum size 
    \usepackage{xcolor} % Allow colors to be defined
    \usepackage{enumerate} % Needed for markdown enumerations to work
    \usepackage{geometry} % Used to adjust the document margins
    \usepackage{amsmath} % Equations
    \usepackage{amssymb} % Equations
    \usepackage{textcomp} % defines textquotesingle
    % Hack from http://tex.stackexchange.com/a/47451/13684:
    \AtBeginDocument{%
        \def\PYZsq{\textquotesingle}% Upright quotes in Pygmentized code
    }
    \usepackage{upquote} % Upright quotes for verbatim code
    \usepackage{eurosym} % defines \euro
    \usepackage[mathletters]{ucs} % Extended unicode (utf-8) support
    \usepackage[utf8x]{inputenc} % Allow utf-8 characters in the tex document
    \usepackage{fancyvrb} % verbatim replacement that allows latex
    \usepackage{grffile} % extends the file name processing of package graphics 
                         % to support a larger range 
    % The hyperref package gives us a pdf with properly built
    % internal navigation ('pdf bookmarks' for the table of contents,
    % internal cross-reference links, web links for URLs, etc.)
    \usepackage{hyperref}
    \usepackage{longtable} % longtable support required by pandoc >1.10
    \usepackage{booktabs}  % table support for pandoc > 1.12.2
    \usepackage[inline]{enumitem} % IRkernel/repr support (it uses the enumerate* environment)
    \usepackage[normalem]{ulem} % ulem is needed to support strikethroughs (\sout)
                                % normalem makes italics be italics, not underlines
    

    
    
    % Colors for the hyperref package
    \definecolor{urlcolor}{rgb}{0,.145,.698}
    \definecolor{linkcolor}{rgb}{.71,0.21,0.01}
    \definecolor{citecolor}{rgb}{.12,.54,.11}

    % ANSI colors
    \definecolor{ansi-black}{HTML}{3E424D}
    \definecolor{ansi-black-intense}{HTML}{282C36}
    \definecolor{ansi-red}{HTML}{E75C58}
    \definecolor{ansi-red-intense}{HTML}{B22B31}
    \definecolor{ansi-green}{HTML}{00A250}
    \definecolor{ansi-green-intense}{HTML}{007427}
    \definecolor{ansi-yellow}{HTML}{DDB62B}
    \definecolor{ansi-yellow-intense}{HTML}{B27D12}
    \definecolor{ansi-blue}{HTML}{208FFB}
    \definecolor{ansi-blue-intense}{HTML}{0065CA}
    \definecolor{ansi-magenta}{HTML}{D160C4}
    \definecolor{ansi-magenta-intense}{HTML}{A03196}
    \definecolor{ansi-cyan}{HTML}{60C6C8}
    \definecolor{ansi-cyan-intense}{HTML}{258F8F}
    \definecolor{ansi-white}{HTML}{C5C1B4}
    \definecolor{ansi-white-intense}{HTML}{A1A6B2}

    % commands and environments needed by pandoc snippets
    % extracted from the output of `pandoc -s`
    \providecommand{\tightlist}{%
      \setlength{\itemsep}{0pt}\setlength{\parskip}{0pt}}
    \DefineVerbatimEnvironment{Highlighting}{Verbatim}{commandchars=\\\{\}}
    % Add ',fontsize=\small' for more characters per line
    \newenvironment{Shaded}{}{}
    \newcommand{\KeywordTok}[1]{\textcolor[rgb]{0.00,0.44,0.13}{\textbf{{#1}}}}
    \newcommand{\DataTypeTok}[1]{\textcolor[rgb]{0.56,0.13,0.00}{{#1}}}
    \newcommand{\DecValTok}[1]{\textcolor[rgb]{0.25,0.63,0.44}{{#1}}}
    \newcommand{\BaseNTok}[1]{\textcolor[rgb]{0.25,0.63,0.44}{{#1}}}
    \newcommand{\FloatTok}[1]{\textcolor[rgb]{0.25,0.63,0.44}{{#1}}}
    \newcommand{\CharTok}[1]{\textcolor[rgb]{0.25,0.44,0.63}{{#1}}}
    \newcommand{\StringTok}[1]{\textcolor[rgb]{0.25,0.44,0.63}{{#1}}}
    \newcommand{\CommentTok}[1]{\textcolor[rgb]{0.38,0.63,0.69}{\textit{{#1}}}}
    \newcommand{\OtherTok}[1]{\textcolor[rgb]{0.00,0.44,0.13}{{#1}}}
    \newcommand{\AlertTok}[1]{\textcolor[rgb]{1.00,0.00,0.00}{\textbf{{#1}}}}
    \newcommand{\FunctionTok}[1]{\textcolor[rgb]{0.02,0.16,0.49}{{#1}}}
    \newcommand{\RegionMarkerTok}[1]{{#1}}
    \newcommand{\ErrorTok}[1]{\textcolor[rgb]{1.00,0.00,0.00}{\textbf{{#1}}}}
    \newcommand{\NormalTok}[1]{{#1}}
    
    % Additional commands for more recent versions of Pandoc
    \newcommand{\ConstantTok}[1]{\textcolor[rgb]{0.53,0.00,0.00}{{#1}}}
    \newcommand{\SpecialCharTok}[1]{\textcolor[rgb]{0.25,0.44,0.63}{{#1}}}
    \newcommand{\VerbatimStringTok}[1]{\textcolor[rgb]{0.25,0.44,0.63}{{#1}}}
    \newcommand{\SpecialStringTok}[1]{\textcolor[rgb]{0.73,0.40,0.53}{{#1}}}
    \newcommand{\ImportTok}[1]{{#1}}
    \newcommand{\DocumentationTok}[1]{\textcolor[rgb]{0.73,0.13,0.13}{\textit{{#1}}}}
    \newcommand{\AnnotationTok}[1]{\textcolor[rgb]{0.38,0.63,0.69}{\textbf{\textit{{#1}}}}}
    \newcommand{\CommentVarTok}[1]{\textcolor[rgb]{0.38,0.63,0.69}{\textbf{\textit{{#1}}}}}
    \newcommand{\VariableTok}[1]{\textcolor[rgb]{0.10,0.09,0.49}{{#1}}}
    \newcommand{\ControlFlowTok}[1]{\textcolor[rgb]{0.00,0.44,0.13}{\textbf{{#1}}}}
    \newcommand{\OperatorTok}[1]{\textcolor[rgb]{0.40,0.40,0.40}{{#1}}}
    \newcommand{\BuiltInTok}[1]{{#1}}
    \newcommand{\ExtensionTok}[1]{{#1}}
    \newcommand{\PreprocessorTok}[1]{\textcolor[rgb]{0.74,0.48,0.00}{{#1}}}
    \newcommand{\AttributeTok}[1]{\textcolor[rgb]{0.49,0.56,0.16}{{#1}}}
    \newcommand{\InformationTok}[1]{\textcolor[rgb]{0.38,0.63,0.69}{\textbf{\textit{{#1}}}}}
    \newcommand{\WarningTok}[1]{\textcolor[rgb]{0.38,0.63,0.69}{\textbf{\textit{{#1}}}}}
    
    
    % Define a nice break command that doesn't care if a line doesn't already
    % exist.
    \def\br{\hspace*{\fill} \\* }
    % Math Jax compatability definitions
    \def\gt{>}
    \def\lt{<}
    % Document parameters
    \title{midterm\_v02}
    
    
    

    % Pygments definitions
    
\makeatletter
\def\PY@reset{\let\PY@it=\relax \let\PY@bf=\relax%
    \let\PY@ul=\relax \let\PY@tc=\relax%
    \let\PY@bc=\relax \let\PY@ff=\relax}
\def\PY@tok#1{\csname PY@tok@#1\endcsname}
\def\PY@toks#1+{\ifx\relax#1\empty\else%
    \PY@tok{#1}\expandafter\PY@toks\fi}
\def\PY@do#1{\PY@bc{\PY@tc{\PY@ul{%
    \PY@it{\PY@bf{\PY@ff{#1}}}}}}}
\def\PY#1#2{\PY@reset\PY@toks#1+\relax+\PY@do{#2}}

\expandafter\def\csname PY@tok@w\endcsname{\def\PY@tc##1{\textcolor[rgb]{0.73,0.73,0.73}{##1}}}
\expandafter\def\csname PY@tok@c\endcsname{\let\PY@it=\textit\def\PY@tc##1{\textcolor[rgb]{0.25,0.50,0.50}{##1}}}
\expandafter\def\csname PY@tok@cp\endcsname{\def\PY@tc##1{\textcolor[rgb]{0.74,0.48,0.00}{##1}}}
\expandafter\def\csname PY@tok@k\endcsname{\let\PY@bf=\textbf\def\PY@tc##1{\textcolor[rgb]{0.00,0.50,0.00}{##1}}}
\expandafter\def\csname PY@tok@kp\endcsname{\def\PY@tc##1{\textcolor[rgb]{0.00,0.50,0.00}{##1}}}
\expandafter\def\csname PY@tok@kt\endcsname{\def\PY@tc##1{\textcolor[rgb]{0.69,0.00,0.25}{##1}}}
\expandafter\def\csname PY@tok@o\endcsname{\def\PY@tc##1{\textcolor[rgb]{0.40,0.40,0.40}{##1}}}
\expandafter\def\csname PY@tok@ow\endcsname{\let\PY@bf=\textbf\def\PY@tc##1{\textcolor[rgb]{0.67,0.13,1.00}{##1}}}
\expandafter\def\csname PY@tok@nb\endcsname{\def\PY@tc##1{\textcolor[rgb]{0.00,0.50,0.00}{##1}}}
\expandafter\def\csname PY@tok@nf\endcsname{\def\PY@tc##1{\textcolor[rgb]{0.00,0.00,1.00}{##1}}}
\expandafter\def\csname PY@tok@nc\endcsname{\let\PY@bf=\textbf\def\PY@tc##1{\textcolor[rgb]{0.00,0.00,1.00}{##1}}}
\expandafter\def\csname PY@tok@nn\endcsname{\let\PY@bf=\textbf\def\PY@tc##1{\textcolor[rgb]{0.00,0.00,1.00}{##1}}}
\expandafter\def\csname PY@tok@ne\endcsname{\let\PY@bf=\textbf\def\PY@tc##1{\textcolor[rgb]{0.82,0.25,0.23}{##1}}}
\expandafter\def\csname PY@tok@nv\endcsname{\def\PY@tc##1{\textcolor[rgb]{0.10,0.09,0.49}{##1}}}
\expandafter\def\csname PY@tok@no\endcsname{\def\PY@tc##1{\textcolor[rgb]{0.53,0.00,0.00}{##1}}}
\expandafter\def\csname PY@tok@nl\endcsname{\def\PY@tc##1{\textcolor[rgb]{0.63,0.63,0.00}{##1}}}
\expandafter\def\csname PY@tok@ni\endcsname{\let\PY@bf=\textbf\def\PY@tc##1{\textcolor[rgb]{0.60,0.60,0.60}{##1}}}
\expandafter\def\csname PY@tok@na\endcsname{\def\PY@tc##1{\textcolor[rgb]{0.49,0.56,0.16}{##1}}}
\expandafter\def\csname PY@tok@nt\endcsname{\let\PY@bf=\textbf\def\PY@tc##1{\textcolor[rgb]{0.00,0.50,0.00}{##1}}}
\expandafter\def\csname PY@tok@nd\endcsname{\def\PY@tc##1{\textcolor[rgb]{0.67,0.13,1.00}{##1}}}
\expandafter\def\csname PY@tok@s\endcsname{\def\PY@tc##1{\textcolor[rgb]{0.73,0.13,0.13}{##1}}}
\expandafter\def\csname PY@tok@sd\endcsname{\let\PY@it=\textit\def\PY@tc##1{\textcolor[rgb]{0.73,0.13,0.13}{##1}}}
\expandafter\def\csname PY@tok@si\endcsname{\let\PY@bf=\textbf\def\PY@tc##1{\textcolor[rgb]{0.73,0.40,0.53}{##1}}}
\expandafter\def\csname PY@tok@se\endcsname{\let\PY@bf=\textbf\def\PY@tc##1{\textcolor[rgb]{0.73,0.40,0.13}{##1}}}
\expandafter\def\csname PY@tok@sr\endcsname{\def\PY@tc##1{\textcolor[rgb]{0.73,0.40,0.53}{##1}}}
\expandafter\def\csname PY@tok@ss\endcsname{\def\PY@tc##1{\textcolor[rgb]{0.10,0.09,0.49}{##1}}}
\expandafter\def\csname PY@tok@sx\endcsname{\def\PY@tc##1{\textcolor[rgb]{0.00,0.50,0.00}{##1}}}
\expandafter\def\csname PY@tok@m\endcsname{\def\PY@tc##1{\textcolor[rgb]{0.40,0.40,0.40}{##1}}}
\expandafter\def\csname PY@tok@gh\endcsname{\let\PY@bf=\textbf\def\PY@tc##1{\textcolor[rgb]{0.00,0.00,0.50}{##1}}}
\expandafter\def\csname PY@tok@gu\endcsname{\let\PY@bf=\textbf\def\PY@tc##1{\textcolor[rgb]{0.50,0.00,0.50}{##1}}}
\expandafter\def\csname PY@tok@gd\endcsname{\def\PY@tc##1{\textcolor[rgb]{0.63,0.00,0.00}{##1}}}
\expandafter\def\csname PY@tok@gi\endcsname{\def\PY@tc##1{\textcolor[rgb]{0.00,0.63,0.00}{##1}}}
\expandafter\def\csname PY@tok@gr\endcsname{\def\PY@tc##1{\textcolor[rgb]{1.00,0.00,0.00}{##1}}}
\expandafter\def\csname PY@tok@ge\endcsname{\let\PY@it=\textit}
\expandafter\def\csname PY@tok@gs\endcsname{\let\PY@bf=\textbf}
\expandafter\def\csname PY@tok@gp\endcsname{\let\PY@bf=\textbf\def\PY@tc##1{\textcolor[rgb]{0.00,0.00,0.50}{##1}}}
\expandafter\def\csname PY@tok@go\endcsname{\def\PY@tc##1{\textcolor[rgb]{0.53,0.53,0.53}{##1}}}
\expandafter\def\csname PY@tok@gt\endcsname{\def\PY@tc##1{\textcolor[rgb]{0.00,0.27,0.87}{##1}}}
\expandafter\def\csname PY@tok@err\endcsname{\def\PY@bc##1{\setlength{\fboxsep}{0pt}\fcolorbox[rgb]{1.00,0.00,0.00}{1,1,1}{\strut ##1}}}
\expandafter\def\csname PY@tok@kc\endcsname{\let\PY@bf=\textbf\def\PY@tc##1{\textcolor[rgb]{0.00,0.50,0.00}{##1}}}
\expandafter\def\csname PY@tok@kd\endcsname{\let\PY@bf=\textbf\def\PY@tc##1{\textcolor[rgb]{0.00,0.50,0.00}{##1}}}
\expandafter\def\csname PY@tok@kn\endcsname{\let\PY@bf=\textbf\def\PY@tc##1{\textcolor[rgb]{0.00,0.50,0.00}{##1}}}
\expandafter\def\csname PY@tok@kr\endcsname{\let\PY@bf=\textbf\def\PY@tc##1{\textcolor[rgb]{0.00,0.50,0.00}{##1}}}
\expandafter\def\csname PY@tok@bp\endcsname{\def\PY@tc##1{\textcolor[rgb]{0.00,0.50,0.00}{##1}}}
\expandafter\def\csname PY@tok@fm\endcsname{\def\PY@tc##1{\textcolor[rgb]{0.00,0.00,1.00}{##1}}}
\expandafter\def\csname PY@tok@vc\endcsname{\def\PY@tc##1{\textcolor[rgb]{0.10,0.09,0.49}{##1}}}
\expandafter\def\csname PY@tok@vg\endcsname{\def\PY@tc##1{\textcolor[rgb]{0.10,0.09,0.49}{##1}}}
\expandafter\def\csname PY@tok@vi\endcsname{\def\PY@tc##1{\textcolor[rgb]{0.10,0.09,0.49}{##1}}}
\expandafter\def\csname PY@tok@vm\endcsname{\def\PY@tc##1{\textcolor[rgb]{0.10,0.09,0.49}{##1}}}
\expandafter\def\csname PY@tok@sa\endcsname{\def\PY@tc##1{\textcolor[rgb]{0.73,0.13,0.13}{##1}}}
\expandafter\def\csname PY@tok@sb\endcsname{\def\PY@tc##1{\textcolor[rgb]{0.73,0.13,0.13}{##1}}}
\expandafter\def\csname PY@tok@sc\endcsname{\def\PY@tc##1{\textcolor[rgb]{0.73,0.13,0.13}{##1}}}
\expandafter\def\csname PY@tok@dl\endcsname{\def\PY@tc##1{\textcolor[rgb]{0.73,0.13,0.13}{##1}}}
\expandafter\def\csname PY@tok@s2\endcsname{\def\PY@tc##1{\textcolor[rgb]{0.73,0.13,0.13}{##1}}}
\expandafter\def\csname PY@tok@sh\endcsname{\def\PY@tc##1{\textcolor[rgb]{0.73,0.13,0.13}{##1}}}
\expandafter\def\csname PY@tok@s1\endcsname{\def\PY@tc##1{\textcolor[rgb]{0.73,0.13,0.13}{##1}}}
\expandafter\def\csname PY@tok@mb\endcsname{\def\PY@tc##1{\textcolor[rgb]{0.40,0.40,0.40}{##1}}}
\expandafter\def\csname PY@tok@mf\endcsname{\def\PY@tc##1{\textcolor[rgb]{0.40,0.40,0.40}{##1}}}
\expandafter\def\csname PY@tok@mh\endcsname{\def\PY@tc##1{\textcolor[rgb]{0.40,0.40,0.40}{##1}}}
\expandafter\def\csname PY@tok@mi\endcsname{\def\PY@tc##1{\textcolor[rgb]{0.40,0.40,0.40}{##1}}}
\expandafter\def\csname PY@tok@il\endcsname{\def\PY@tc##1{\textcolor[rgb]{0.40,0.40,0.40}{##1}}}
\expandafter\def\csname PY@tok@mo\endcsname{\def\PY@tc##1{\textcolor[rgb]{0.40,0.40,0.40}{##1}}}
\expandafter\def\csname PY@tok@ch\endcsname{\let\PY@it=\textit\def\PY@tc##1{\textcolor[rgb]{0.25,0.50,0.50}{##1}}}
\expandafter\def\csname PY@tok@cm\endcsname{\let\PY@it=\textit\def\PY@tc##1{\textcolor[rgb]{0.25,0.50,0.50}{##1}}}
\expandafter\def\csname PY@tok@cpf\endcsname{\let\PY@it=\textit\def\PY@tc##1{\textcolor[rgb]{0.25,0.50,0.50}{##1}}}
\expandafter\def\csname PY@tok@c1\endcsname{\let\PY@it=\textit\def\PY@tc##1{\textcolor[rgb]{0.25,0.50,0.50}{##1}}}
\expandafter\def\csname PY@tok@cs\endcsname{\let\PY@it=\textit\def\PY@tc##1{\textcolor[rgb]{0.25,0.50,0.50}{##1}}}

\def\PYZbs{\char`\\}
\def\PYZus{\char`\_}
\def\PYZob{\char`\{}
\def\PYZcb{\char`\}}
\def\PYZca{\char`\^}
\def\PYZam{\char`\&}
\def\PYZlt{\char`\<}
\def\PYZgt{\char`\>}
\def\PYZsh{\char`\#}
\def\PYZpc{\char`\%}
\def\PYZdl{\char`\$}
\def\PYZhy{\char`\-}
\def\PYZsq{\char`\'}
\def\PYZdq{\char`\"}
\def\PYZti{\char`\~}
% for compatibility with earlier versions
\def\PYZat{@}
\def\PYZlb{[}
\def\PYZrb{]}
\makeatother


    % Exact colors from NB
    \definecolor{incolor}{rgb}{0.0, 0.0, 0.5}
    \definecolor{outcolor}{rgb}{0.545, 0.0, 0.0}



    
    % Prevent overflowing lines due to hard-to-break entities
    \sloppy 
    % Setup hyperref package
    \hypersetup{
      breaklinks=true,  % so long urls are correctly broken across lines
      colorlinks=true,
      urlcolor=urlcolor,
      linkcolor=linkcolor,
      citecolor=citecolor,
      }
    % Slightly bigger margins than the latex defaults
    
    \geometry{verbose,tmargin=1in,bmargin=1in,lmargin=1in,rmargin=1in}
    
    

    \begin{document}
    
    
    \maketitle
    
    

    
    \textbf{Name}

\textbf{Rules}

\begin{itemize}
\tightlist
\item
  After finish, you should email to me PDF version with file name
  "firstname\_lastname\_midterm.pdf"
\item
  Any kind of communications are strictly prohibited. But you can use
  book, note, or internet.
\item
  All your answers shall be rigourously justified.
\item
  No python modules shall be imported except those given below
\end{itemize}

    \begin{Verbatim}[commandchars=\\\{\}]
{\color{incolor}In [{\color{incolor}1}]:} \PY{k+kn}{import} \PY{n+nn}{numpy} \PY{k}{as} \PY{n+nn}{np}
        \PY{k+kn}{import} \PY{n+nn}{matplotlib}\PY{n+nn}{.}\PY{n+nn}{pyplot} \PY{k}{as} \PY{n+nn}{plt}
        \PY{k+kn}{from} \PY{n+nn}{scipy}\PY{n+nn}{.}\PY{n+nn}{stats} \PY{k}{import} \PY{n}{norm}
        \PY{k+kn}{from} \PY{n+nn}{scipy}\PY{n+nn}{.}\PY{n+nn}{integrate} \PY{k}{import} \PY{n}{quad}
        \PY{k+kn}{from} \PY{n+nn}{scipy}\PY{n+nn}{.}\PY{n+nn}{optimize} \PY{k}{import} \PY{n}{fmin}
\end{Verbatim}


    \textbf{Problem}

    We consider BSM model with the following parameters:

    \begin{Verbatim}[commandchars=\\\{\}]
{\color{incolor}In [{\color{incolor}2}]:} \PY{l+s+sd}{\PYZsq{}\PYZsq{}\PYZsq{}======parameters=============\PYZsq{}\PYZsq{}\PYZsq{}}
        \PY{n}{s0}\PY{p}{,} \PY{n}{r}\PY{p}{,} \PY{n}{sigma}\PY{p}{,} \PY{n}{T}\PY{p}{,} \PY{n}{K}\PY{p}{,} \PY{n}{otype} \PY{o}{=} \PY{p}{[}\PY{l+m+mf}{30.}\PY{p}{,} \PY{o}{.}\PY{l+m+mi}{02}\PY{p}{,} \PY{l+m+mf}{0.3}\PY{p}{,} \PY{l+m+mf}{1.}\PY{p}{,} \PY{l+m+mf}{31.0}\PY{p}{,} \PY{l+m+mi}{1}\PY{p}{]} \PY{c+c1}{\PYZsh{} spot, rate, vol, maturity, strike, option type}
\end{Verbatim}


    \begin{itemize}
\item
  \begin{enumerate}
  \def\labelenumi{(\Alph{enumi})}
  \setcounter{enumi}{16}
  \tightlist
  \item
    Write for the european call the explicit pricing formula, then find
    call price.
  \end{enumerate}
\end{itemize}

    (your latex here)

    \begin{Verbatim}[commandchars=\\\{\}]
{\color{incolor}In [{\color{incolor}3}]:} \PY{l+s+sd}{\PYZsq{}\PYZsq{}\PYZsq{}========}
        \PY{l+s+sd}{Black\PYZhy{}Scholes\PYZhy{}Merton formula. }
        \PY{l+s+sd}{==========\PYZsq{}\PYZsq{}\PYZsq{}}
        
        \PY{k}{def} \PY{n+nf}{bsm}\PY{p}{(}\PY{n}{s0}\PY{p}{,} \PY{n}{r}\PY{p}{,} \PY{n}{sigma}\PY{p}{,} \PY{n}{T}\PY{p}{,} \PY{n}{K}\PY{p}{,} \PY{n}{otype}\PY{p}{)}\PY{p}{:}
            \PY{k}{pass}
        
        \PY{n+nb}{print}\PY{p}{(}\PY{l+s+s1}{\PYZsq{}}\PY{l+s+s1}{\PYZgt{}\PYZgt{}\PYZgt{}\PYZgt{} your answer below: }\PY{l+s+s1}{\PYZsq{}}\PY{p}{)}
\end{Verbatim}


    \begin{Verbatim}[commandchars=\\\{\}]
>>>> your answer below: 

    \end{Verbatim}

    \begin{itemize}
\item
  \begin{enumerate}
  \def\labelenumi{(\Alph{enumi})}
  \setcounter{enumi}{16}
  \item
  \end{enumerate}

  \begin{itemize}
  \tightlist
  \item
    write Fourier transform \(\phi\) of \(\ln S_T\);
  \item
    write down the call price in terms of \(\phi\);
  \item
    then find call price using the above formula.
  \end{itemize}
\end{itemize}

    (your latex here)

    \begin{Verbatim}[commandchars=\\\{\}]
{\color{incolor}In [{\color{incolor}4}]:} \PY{l+s+sd}{\PYZsq{}\PYZsq{}\PYZsq{}========}
        \PY{l+s+sd}{Black\PYZhy{}Scholes\PYZhy{}Merton price by Fourier transform. }
        \PY{l+s+sd}{==========\PYZsq{}\PYZsq{}\PYZsq{}}
        
        \PY{k}{def} \PY{n+nf}{bsm\PYZus{}fourier}\PY{p}{(}\PY{n}{s0}\PY{p}{,} \PY{n}{r}\PY{p}{,} \PY{n}{sigma}\PY{p}{,} \PY{n}{T}\PY{p}{,} \PY{n}{K}\PY{p}{,} \PY{n}{otype}\PY{p}{)}\PY{p}{:}
            \PY{k}{pass}
        
        \PY{n+nb}{print}\PY{p}{(}\PY{l+s+s1}{\PYZsq{}}\PY{l+s+s1}{\PYZgt{}\PYZgt{}\PYZgt{} your answer below }\PY{l+s+s1}{\PYZsq{}}\PY{p}{)}
\end{Verbatim}


    \begin{Verbatim}[commandchars=\\\{\}]
>>> your answer below 

    \end{Verbatim}

    \begin{itemize}
\item
  \begin{enumerate}
  \def\labelenumi{(\Alph{enumi})}
  \setcounter{enumi}{16}
  \tightlist
  \item
    Prove that call price is increasing with respect to its maturity,
    and demonstrate your result numerically.
  \end{enumerate}
\end{itemize}

    (your latex here)

    \begin{Verbatim}[commandchars=\\\{\}]
{\color{incolor}In [{\color{incolor}5}]:} \PY{l+s+sd}{\PYZsq{}\PYZsq{}\PYZsq{}====your code here===\PYZsq{}\PYZsq{}\PYZsq{}}
        
        \PY{k}{pass}
\end{Verbatim}


    \begin{itemize}
\item
  \begin{enumerate}
  \def\labelenumi{(\Alph{enumi})}
  \setcounter{enumi}{16}
  \tightlist
  \item
    Do you think that call price is also increasing with respect to its
    maturity for Heston model? Please justify your answer.
  \end{enumerate}
\end{itemize}

    (your latex here)

    \begin{itemize}
\item
  \begin{enumerate}
  \def\labelenumi{(\Alph{enumi})}
  \setcounter{enumi}{16}
  \tightlist
  \item
    With the same parameters for GBM above, price the arithmetic asian
    option with \(T = 1\), \(K= 31\), and two periods (i.e. average
    prices of three time instants)
  \end{enumerate}
\end{itemize}

    \begin{Verbatim}[commandchars=\\\{\}]
{\color{incolor}In [{\color{incolor}6}]:} \PY{l+s+sd}{\PYZsq{}\PYZsq{}\PYZsq{}====your code here===\PYZsq{}\PYZsq{}\PYZsq{}}
        
        \PY{k}{pass}
\end{Verbatim}


    \textbf{Problem}

    We consider the call price underlying XYZ stock with spot price \(\$30\)
and interest rate \(2\%\). Below is the current market call prices
corresponding to its maturity \(T=1\) and different strikes:

    \begin{Verbatim}[commandchars=\\\{\}]
{\color{incolor}In [{\color{incolor}7}]:} \PY{n}{s0}\PY{p}{,} \PY{n}{r}\PY{p}{,} \PY{n}{T}\PY{p}{,} \PY{n}{otype} \PY{o}{=} \PY{p}{[}\PY{l+m+mf}{30.}\PY{p}{,} \PY{o}{.}\PY{l+m+mi}{02}\PY{p}{,} \PY{l+m+mf}{1.}\PY{p}{,} \PY{l+m+mi}{1}\PY{p}{]} \PY{c+c1}{\PYZsh{} spot, rate, maturity, option type}
        \PY{n}{strike\PYZus{}arr} \PY{o}{=} \PY{p}{[}\PY{l+m+mf}{28.0}\PY{p}{,} \PY{l+m+mf}{28.5}\PY{p}{,} \PY{l+m+mf}{29.0}\PY{p}{,} \PY{l+m+mf}{29.5}\PY{p}{,} \PY{l+m+mf}{30.0}\PY{p}{,} \PY{l+m+mf}{30.5}\PY{p}{,} \PY{l+m+mf}{31.0}\PY{p}{,} \PY{l+m+mf}{31.5}\PY{p}{,} \PY{l+m+mf}{32.0}\PY{p}{]} \PY{c+c1}{\PYZsh{} strike array}
        \PY{n}{market\PYZus{}call\PYZus{}arr} \PY{o}{=} \PY{p}{[}\PY{l+m+mf}{5.7}\PY{p}{,} \PY{l+m+mf}{5.0}\PY{p}{,} \PY{l+m+mf}{4.5}\PY{p}{,} \PY{l+m+mf}{4.1}\PY{p}{,} \PY{l+m+mf}{3.8}\PY{p}{,} \PY{l+m+mf}{3.6}\PY{p}{,} \PY{l+m+mf}{3.6}\PY{p}{,} \PY{l+m+mf}{3.7}\PY{p}{,} \PY{l+m+mf}{3.9}\PY{p}{]} \PY{c+c1}{\PYZsh{} corresponding market price}
\end{Verbatim}


    \begin{itemize}
\item
  \begin{enumerate}
  \def\labelenumi{(\Alph{enumi})}
  \setcounter{enumi}{16}
  \tightlist
  \item
    Find their implied volatilities and draw its volatility smile
  \end{enumerate}
\end{itemize}

    \begin{Verbatim}[commandchars=\\\{\}]
{\color{incolor}In [{\color{incolor}8}]:} \PY{l+s+sd}{\PYZsq{}\PYZsq{}\PYZsq{}====your code here====\PYZsq{}\PYZsq{}\PYZsq{}}
        
        \PY{k}{pass}
\end{Verbatim}


    \begin{itemize}
\tightlist
\item
  (Q.) Find bsm calibrated volatility to the above data by minimizing
  \(L^2\) norm in the error.
\end{itemize}

    \begin{Verbatim}[commandchars=\\\{\}]
{\color{incolor}In [{\color{incolor}9}]:} \PY{l+s+sd}{\PYZsq{}\PYZsq{}\PYZsq{}====your code here====\PYZsq{}\PYZsq{}\PYZsq{}}
        
        \PY{k}{pass}
\end{Verbatim}


    \textbf{Problem}

    Let \(f:\mathbb R\mapsto \mathbb R\) be a smooth even function
satisfying \(f(0) = 0\). Our objective is to approximate the second
order derivative \(f''(0)\). - (Q.) Prove that \(f'(0) = 0.\) - (Q.)
Ziyue proposes the following estimator for \(f''(0)\): for a step size
\(h\) \[a_{h} = \frac{2f(h)}{h^{2}}.\] Please justify that Ziyue's
estimation has its convergence \(O(h^{2})\). - (Q.) Is there anyway to
improve the above convergence to \(O(h^{4})\) in the form of
\[b_{h} = \frac{c_{1} f(h) + c_{2}f(2h)}{h^{2}}\] for some constants
\(c_{1}\) and \(c_{2}\)? - (Q.) If the above function \(f\) is odd and
other properties remain the same, how do you want to find the \(f''(0)\)
efficiently?

    \textbf{Soln} (your latex input here)

    \textbf{Problem}

    Thomas proposed the following estimator for the value \(\pi\).
\[\hat \pi = c \cdot I(X_1^2 + X_2^2< 1/2)\] where \(X_1\) and \(X_2\)
are two independent uniform random variables on \((-1,1)\).

\begin{itemize}
\item
  \begin{enumerate}
  \def\labelenumi{(\Alph{enumi})}
  \setcounter{enumi}{16}
  \tightlist
  \item
    find the constant \(c\) so that \(\hat \pi\) is unbiased estimator
    to \(\pi\);
  \end{enumerate}
\item
  \begin{enumerate}
  \def\labelenumi{(\Alph{enumi})}
  \setcounter{enumi}{16}
  \tightlist
  \item
    for \(c?\) in your above answer, what is its MSE?
  \end{enumerate}
\end{itemize}

    (your latex here)

    \begin{itemize}
\item
  \begin{enumerate}
  \def\labelenumi{(\Alph{enumi})}
  \setcounter{enumi}{16}
  \tightlist
  \item
    can you propose a better mc estimator with smaller MSE?
  \end{enumerate}
\end{itemize}

    \textbf{Soln} (your latex here)

    \begin{itemize}
\item
  \begin{enumerate}
  \def\labelenumi{(\Alph{enumi})}
  \setcounter{enumi}{16}
  \tightlist
  \item
    can you demonstrate the theoretical MSE obtained in your algorithm
    above numerically via python code?
  \end{enumerate}
\end{itemize}

    \begin{Verbatim}[commandchars=\\\{\}]
{\color{incolor}In [{\color{incolor}10}]:} \PY{l+s+sd}{\PYZsq{}\PYZsq{}\PYZsq{}====== your code here =======\PYZsq{}\PYZsq{}\PYZsq{}}
         \PY{k}{pass}
\end{Verbatim}


    \textbf{Problem}

    \begin{itemize}
\item
  \begin{enumerate}
  \def\labelenumi{(\Alph{enumi})}
  \setcounter{enumi}{16}
  \tightlist
  \item
    find the conditional probability of a random varialbe \(Z\) in
    \(B_{1/2}\), given that \(Z\) is uniformly distributed on \(B_1\).
  \end{enumerate}
\end{itemize}

    \textbf{Soln}(your latex here)

    \begin{itemize}
\item
  \begin{enumerate}
  \def\labelenumi{(\Alph{enumi})}
  \setcounter{enumi}{16}
  \tightlist
  \item
    Demonstrate your answer numerically by doing so:
  \end{enumerate}

  \begin{itemize}
  \tightlist
  \item
    Generate 1000 samplings uniformly from \(B_1\);
  \item
    count the points in \(B_{1/2}\).
  \item
    find the ratio of number of points in the small ball and the big
    ball.
  \end{itemize}
\end{itemize}

    \begin{Verbatim}[commandchars=\\\{\}]
{\color{incolor}In [{\color{incolor}11}]:} \PY{l+s+sd}{\PYZsq{}\PYZsq{}\PYZsq{}===== your code here======\PYZsq{}\PYZsq{}\PYZsq{}}
         \PY{n}{num\PYZus{}sample} \PY{o}{=} \PY{l+m+mi}{1000}
         \PY{n}{sample\PYZus{}arr} \PY{o}{=} \PY{n}{np}\PY{o}{.}\PY{n}{zeros}\PY{p}{(}\PY{p}{[}\PY{l+m+mi}{2}\PY{p}{,}\PY{n}{num\PYZus{}sample}\PY{p}{]}\PY{p}{)}
         
         \PY{k}{pass}
\end{Verbatim}


    \begin{itemize}
\item
  \begin{enumerate}
  \def\labelenumi{(\Alph{enumi})}
  \setcounter{enumi}{16}
  \tightlist
  \item
    Justify your random sampling in the previous question
  \end{enumerate}
\end{itemize}

    \textbf{soln} (your latex here)


    % Add a bibliography block to the postdoc
    
    
    
    \end{document}
